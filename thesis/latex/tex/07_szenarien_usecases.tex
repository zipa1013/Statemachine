\chapter{Szenarien und Use-Cases}\label{chap:szenarien}

\noindent
Dieses Kapitel beschreibt die verschiedenen Szenarien und Use-Cases, die für die Simulation und Bewertung der E/E-Architektur verwendet werden. Die Szenarien umfassen nominale Betriebsbedingungen, Stress-Szenarien, Fehlerszenarien und Design-Varianten. Eine umfassende Szenario-Abdeckung ist essentiell, um die Robustheit und Zuverlässigkeit der Architektur unter verschiedenen Bedingungen zu bewerten \cite{simulation_automotive, validation_verification}.

Die Definition und Durchführung von Szenarien folgt etablierten Methoden der Fahrzeugentwicklung \cite{automated_driving_systems} und berücksichtigt die spezifischen Anforderungen von Nutzfahrzeugen \cite{reif_ee_architektur}. Jedes Szenario wird detailliert beschrieben, parametriert und mit erwarteten Ergebnissen versehen.

\section{Nominal-Szenarien}

Nominal-Szenarien beschreiben den normalen Betrieb der Architektur unter typischen Bedingungen. Diese Szenarien dienen als Baseline für die Bewertung der Architektur-Performance und bilden die Grundlage für den Vergleich mit Stress- und Fehlerszenarien.

\subsection{Standard-Betrieb}

Nominal-Szenarien beschreiben den normalen Betrieb der Architektur unter typischen Bedingungen:

\begin{itemize}
  \item \textbf{Lastverteilung}: Normale CPU/GPU/Netzwerk-Auslastung
    \begin{itemize}
      \item CPU-Auslastung: 40-60\% für typische ECUs
      \item GPU-Auslastung: 30-50\% für Perzeptions-Tasks
      \item Netzwerk-Auslastung: 30-50\% für Ethernet-Links
      \item Keine Überlastung, ausreichend Reserve für Spitzenlasten
    \end{itemize}
  
  \item \textbf{Timing}: Alle Frames und Tasks innerhalb ihrer Periodizitäten
    \begin{itemize}
      \item Alle Tasks erfüllen ihre Deadlines
      \item E2E-Latenzen innerhalb der Budgets
      \item Jitter innerhalb akzeptabler Grenzen (< 10\% der Periodizität)
      \item Keine Deadline-Misses
    \end{itemize}
  
  \item \textbf{Kommunikation}: Keine Paketverluste, normale Latenzen
    \begin{itemize}
      \item Paketverlustrate: < $10^{-6}$ für sicherheitskritische Frames
      \item Netzwerk-Latenzen: Innerhalb der Budgets (typisch 1-5 ms für TSN)
      \item Keine Queue-Überläufe
      \item Stabile Kommunikation ohne Retransmissions
    \end{itemize}
  
  \item \textbf{Energie}: Standard-Energieverbrauch
    \begin{itemize}
      \item Energieverbrauch entsprechend Lastprofil
      \item Power-States entsprechend Aktivität
      \item Keine unerwarteten Power-Spikes
    \end{itemize}
\end{itemize}

\subsection{Typische Fahrzyklen}

Verschiedene Fahrzyklen werden modelliert, um unterschiedliche Betriebsbedingungen abzudecken. Jeder Fahrzyklus hat charakteristische Lastprofile und Timing-Anforderungen:

\begin{itemize}
  \item \textbf{Stadtverkehr}: Häufige Starts/Stopps, niedrige Geschwindigkeiten
    \begin{itemize}
      \item \textbf{Charakteristik}: Viele Brems- und Beschleunigungsvorgänge, häufige Spurwechsel
      \item \textbf{Sensoren (Bosch)}:
        \begin{itemize}
          \item 8x Bosch 8MP Multifunktionskameras: Hohe Auflösung für präzise Objekterkennung in komplexen Szenen
          \item 8x Bosch Long-Range-Radar: Für Geschwindigkeits- und Abstandsmessung
          \item 4x Bosch High-Resolution-LiDAR: Für präzise 3D-Umgebungserfassung
          \item 16x Ultraschall-Sensoren: Für Nahbereichserkennung
        \end{itemize}
      \item \textbf{Rechenplattform (NVIDIA DRIVE Thor)}:
        \begin{itemize}
          \item GPU-Last: 60-70\% für Perzeptions-Tasks (YOLOv8 auf 8MP Kameras)
          \item CPU-Last: 40-50\% für Planung und Regelung
          \item Echtzeit-Verarbeitung von bis zu 12 Kameras gleichzeitig
        \end{itemize}
      \item \textbf{Lastprofil}: 
        \begin{itemize}
          \item Perzeption: Hohe Aktivität (viele Objekte, Fußgänger, Radfahrer)
          \item Planung: Häufige Replanung erforderlich
          \item Regelung: Häufige Lenk- und Bremskommandos
        \end{itemize}
      \item \textbf{Netzwerk}: Moderate bis hohe Last durch viele Sensor-Updates (Ethernet 2.5G für Kameras, 10G für LiDAR)
      \item \textbf{Energie}: Variabler Verbrauch durch häufige Beschleunigungen, DRIVE Thor: 80-120 W
      \item \textbf{Beispiel-Dauer}: 30 Minuten simulierter Stadtverkehr
    \end{itemize}
  
  \item \textbf{Autobahn}: Hohe Geschwindigkeiten, konstante Last
    \begin{itemize}
      \item \textbf{Charakteristik}: Hohe konstante Geschwindigkeit, wenig Spurwechsel
      \item \textbf{Lastprofil}:
        \begin{itemize}
          \item Perzeption: Moderate Aktivität (weniger Objekte, aber höhere Geschwindigkeit)
          \item Planung: Weniger Replanung erforderlich
          \item Regelung: Konstante Lenkkommandos für Spurhaltung
        \end{itemize}
      \item \textbf{Netzwerk}: Stabile Last, weniger Variationen
      \item \textbf{Energie}: Relativ konstanter Verbrauch
      \item \textbf{Beispiel-Dauer}: 60 Minuten simulierter Autobahnfahrt
    \end{itemize}
  
  \item \textbf{Landstraße}: Variable Geschwindigkeiten, moderate Last
    \begin{itemize}
      \item \textbf{Charakteristik}: Variable Geschwindigkeiten, Kurven, Überholvorgänge
      \item \textbf{Lastprofil}:
        \begin{itemize}
          \item Perzeption: Moderate bis hohe Aktivität
          \item Planung: Periodische Replanung
          \item Regelung: Variable Lenk- und Bremskommandos
        \end{itemize}
      \item \textbf{Netzwerk}: Variable Last
      \item \textbf{Energie}: Variabler Verbrauch
    \end{itemize}
  
  \item \textbf{Parkplatz}: Niedrige Last, viele Sensoren aktiv
    \begin{itemize}
      \item \textbf{Charakteristik}: Niedrige Geschwindigkeit, viele Objekte, enge Räume
      \item \textbf{Lastprofil}:
        \begin{itemize}
          \item Perzeption: Sehr hohe Aktivität (viele Objekte, enge Räume)
          \item Planung: Häufige Replanung für enge Manöver
          \item Regelung: Präzise Lenk- und Bremskommandos
        \end{itemize}
      \item \textbf{Netzwerk}: Hohe Last durch viele Sensor-Updates (Ultraschall, Kameras)
      \item \textbf{Energie}: Niedriger Verbrauch (niedrige Geschwindigkeit), aber hohe Rechenlast
    \end{itemize}
\end{itemize}

\begin{table}[h]
  \centering
  \caption{Vergleich typischer Fahrzyklen}
  \begin{tabular}{lllll}
    \toprule
    Fahrzyklus & CPU-Last & Netzwerk-Last & E2E-Latenz & Energie \\
    \midrule
    Stadtverkehr & 55-65\% & 40-50\% & 80-95 ms & Variabel \\
    Autobahn & 45-55\% & 30-40\% & 70-85 ms & Konstant \\
    Landstraße & 50-60\% & 35-45\% & 75-90 ms & Variabel \\
    Parkplatz & 60-70\% & 50-60\% & 85-100 ms & Niedrig \\
    \bottomrule
  \end{tabular}
  \label{tab:fahrzyklen_vergleich}
\end{table}

\section{Stress-Szenarien}

\subsection{Stau-Szenario}

Ein Stau-Szenario simuliert hohe Last durch viele Objekte:

\begin{itemize}
  \item \textbf{Perzeption}: Viele Objekte in der Szene (50+ Fahrzeuge, Fußgänger)
  \item \textbf{Netzwerk}: Hohe Datenraten durch viele Sensor-Streams
  \item \textbf{CPU}: Hohe Auslastung durch komplexe Objekterkennung
  \item \textbf{Ziel}: Prüfung, ob Deadlines auch unter hoher Last eingehalten werden
\end{itemize}

\subsection{Wetter-Szenario}

Schlechte Wetterbedingungen erhöhen die Verarbeitungslast:

\begin{itemize}
  \item \textbf{Regen/Schnee}: Reduzierte Sicht, mehr Bildverarbeitung erforderlich
  \item \textbf{Nebel}: Zusätzliche Filterung und Fusion erforderlich
  \item \textbf{Starke Sonne}: Blendung, erhöhte Bildverarbeitung
  \item \textbf{Ziel}: Prüfung der Robustheit unter schwierigen Bedingungen
\end{itemize}

\subsection{Extreme Last-Szenarien}

\begin{itemize}
  \item \textbf{Maximale Datenrate}: Alle Sensoren mit maximaler Framerate
  \item \textbf{Maximale CPU-Last}: Alle Tasks gleichzeitig aktiv
  \item \textbf{Maximale Netzwerk-Auslastung}: Alle Links nahe 100\% Auslastung
  \item \textbf{Ziel}: Identifikation von Bottlenecks und Grenzen
\end{itemize}

\section{Fehlerszenarien}

\subsection{ECU-Ausfall}

\subsubsection{Einzelner ECU-Ausfall}

\begin{itemize}
  \item \textbf{Ausfallzeitpunkt}: Zufällig während der Simulation
  \item \textbf{Ausfallmodus}: Sofortiger Ausfall oder Graduelle Degradation
  \item \textbf{Auswirkungen}: 
    \begin{itemize}
      \item Verlust von Funktionen auf dem ausgefallenen ECU
      \item Erhöhte Last auf anderen ECUs (falls Redundanz vorhanden)
      \item Kommunikationsausfälle
    \end{itemize}
  \item \textbf{Ziel}: Prüfung der Fehlertoleranz und Redundanz-Mechanismen
\end{itemize}

\subsubsection{Mehrfach-Ausfälle}

\begin{itemize}
  \item \textbf{Simultane Ausfälle}: Mehrere ECUs fallen gleichzeitig aus
  \item \textbf{Kaskadierende Ausfälle}: Ein Ausfall führt zu weiteren Ausfällen
  \item \textbf{Ziel}: Prüfung der System-Robustheit
\end{itemize}

\subsection{Link-Ausfall}

\subsubsection{Einzelner Link-Ausfall}

\begin{itemize}
  \item \textbf{Link-Typen}: Ethernet, CAN, LIN
  \item \textbf{Ausfallmodi}: 
    \begin{itemize}
      \item Kompletter Ausfall (Kabelbruch)
      \item Degradierte Bandbreite (Störung)
      \item Erhöhte Latenz (Überlastung)
    \end{itemize}
  \item \textbf{Auswirkungen}: 
    \begin{itemize}
      \item Kommunikationsausfälle
      \item Erhöhte Latenzen über alternative Routen
      \item Paketverluste
    \end{itemize}
\end{itemize}

\subsubsection{Redundante Pfade}

\begin{itemize}
  \item \textbf{PRP/HSR}: Parallel Redundancy Protocol / High-availability Seamless Redundancy
  \item \textbf{Dual-Path}: Zwei unabhängige Kommunikationspfade
  \item \textbf{Ziel}: Prüfung der Redundanz-Mechanismen
\end{itemize}

\subsection{Switch-Queue-Überlauf}

\subsubsection{Überlastung}

\begin{itemize}
  \item \textbf{Ursache}: Zu viele Frames für verfügbare Bandbreite
  \item \textbf{Auswirkung}: Queue-Überlauf, Paketverluste
  \item \textbf{Ziel}: Identifikation von Bandbreiten-Bottlenecks
\end{itemize}

\subsubsection{Priorisierung}

\begin{itemize}
  \item \textbf{TSN-Prioritäten}: Prüfung, ob Priorisierung funktioniert
  \item \textbf{Deadline-Misses}: Prüfung, ob kritische Frames bevorzugt werden
\end{itemize}

\subsection{Clock-Drift}

\subsubsection{Zeitsynchronisation}

\begin{itemize}
  \item \textbf{TSN-Synchronisation}: gPTP (IEEE 802.1AS) Clock-Drift
  \item \textbf{Auswirkung}: Timing-Fehler, Jitter-Erhöhung
  \item \textbf{Ziel}: Prüfung der Robustheit gegen Synchronisations-Fehler
\end{itemize}

\section{Design-Varianten}

\subsection{Redundanz-Varianten}

\subsubsection{Verschiedene Redundanz-Grade}

\begin{itemize}
  \item \textbf{Keine Redundanz}: Single-Point-of-Failure
  \item \textbf{2-fach Redundanz}: 1-out-of-2 Voting
  \item \textbf{3-fach Redundanz}: 2-out-of-3 Voting
  \item \textbf{Ziel}: Vergleich von Verfügbarkeit vs. Kosten
\end{itemize}

\subsubsection{Redundanz-Platzierung}

\begin{itemize}
  \item \textbf{ECU-Redundanz}: Redundante ECUs
  \item \textbf{Link-Redundanz}: Redundante Kommunikationspfade
  \item \textbf{Software-Redundanz}: Redundante Software-Komponenten
\end{itemize}

\subsection{Bandbreiten-Varianten}

\subsubsection{Verschiedene Ethernet-Geschwindigkeiten}

\begin{itemize}
  \item \textbf{1 GbE}: Standard-Geschwindigkeit
  \item \textbf{2.5 GbE}: Erhöhte Geschwindigkeit
  \item \textbf{10 GbE}: Hohe Geschwindigkeit
  \item \textbf{Ziel}: Kosten-Nutzen-Analyse
\end{itemize}

\subsubsection{Bandbreiten-Allokation}

\begin{itemize}
  \item \textbf{Gleichmäßige Verteilung}: Alle Links gleiche Bandbreite
  \item \textbf{Priorisierte Allokation}: Kritische Links erhalten mehr Bandbreite
  \item \textbf{Dynamische Allokation}: Bandbreite wird dynamisch zugewiesen
\end{itemize}

\subsection{Scheduling-Varianten}

\subsubsection{Verschiedene Scheduling-Policies}

\begin{itemize}
  \item \textbf{Fixed Priority}: Feste Prioritäten
  \item \textbf{Earliest Deadline First}: Dynamische Prioritäten
  \item \textbf{Time-Triggered}: Zeitgesteuertes Scheduling
  \item \textbf{Ziel}: Vergleich von Scheduling-Strategien
\end{itemize}

\subsubsection{Prioritäts-Zuordnung}

\begin{itemize}
  \item \textbf{Deadline-basiert}: Priorität basierend auf Deadline
  \item \textbf{ASIL-basiert}: Priorität basierend auf Safety-Level
  \item \textbf{Hybrid}: Kombination verschiedener Kriterien
\end{itemize}

\section{Systematische Variation}

\subsection{Parameter-Sweeps}

Systematische Variation von Parametern:

\begin{itemize}
  \item \textbf{Framegröße}: $\pm 20\%$ Variation
  \item \textbf{Periodizität}: $\pm 10\%$ Variation
  \item \textbf{WCET}: $\pm 15\%$ Variation
  \item \textbf{Bandbreite}: $\pm 25\%$ Variation
\end{itemize}

\subsection{Design-Space-Exploration}

\begin{itemize}
  \item \textbf{Multi-Objective-Optimierung}: Trade-offs zwischen Latenz, Kosten, Energie
  \item \textbf{Pareto-Front}: Identifikation optimaler Design-Punkte
  \item \textbf{Sensitivitäts-Analyse}: Identifikation kritischer Parameter
\end{itemize}

\section{Use-Case-spezifische Szenarien}

\subsection{Autonomes Fahren (L3/L4)}

\begin{itemize}
  \item \textbf{Highway-Pilot}: Autobahn-Autopilot
  \item \textbf{Urban-Pilot}: Stadt-Autopilot
  \item \textbf{Parking}: Automatisches Einparken
  \item \textbf{Anforderungen}: 
    \begin{itemize}
      \item E2E-Latenz < 100 ms
      \item ASIL D für kritische Funktionen
      \item Hohe Verfügbarkeit
    \end{itemize}
\end{itemize}

\subsection{Fahrerassistenz (L2)}

\begin{itemize}
  \item \textbf{ACC}: Adaptive Cruise Control
  \item \textbf{LKA}: Lane Keeping Assist
  \item \textbf{AEB}: Automatic Emergency Braking
  \item \textbf{Anforderungen}: 
    \begin{itemize}
      \item E2E-Latenz < 150 ms
      \item ASIL B/C für kritische Funktionen
    \end{itemize}
\end{itemize}

\subsection{Fracht-/Laderaum-Management}

\begin{itemize}
  \item \textbf{Temperatur-Überwachung}: Kühlung für Fracht
  \item \textbf{Tür-Überwachung}: Sicherheit für Ladung
  \item \textbf{Gewichts-Überwachung}: Lastverteilung
  \item \textbf{Anforderungen}: 
    \begin{itemize}
      \item Kontinuierliche Überwachung
      \item Niedrige Latenz für Alarme
      \item Energieeffizienz
    \end{itemize}
\end{itemize}

\section{Erweiterte Szenario-Beispiele}

Dieser Abschnitt präsentiert detaillierte Beispiele für komplexe Szenarien, die in der Simulation verwendet werden.

\subsection{Beispiel: Vollständiger Stadtverkehr-Zyklus}

Dieses Beispiel zeigt einen vollständigen Stadtverkehr-Zyklus mit allen Phasen und Übergängen.

\subsubsection{Phasen des Zyklus}

Der Zyklus umfasst mehrere Phasen:

\begin{enumerate}
  \item \textbf{Startphase} (0-5 min):
    \begin{itemize}
      \item Fahrzeugstart, Systeminitialisierung
      \item Alle Sensoren aktivieren
      \item Kalibrierung der Sensoren
      \item Netzwerk-Synchronisation (gPTP)
      \item Last: Niedrig (Initialisierung)
    \end{itemize}
  
  \item \textbf{Fahrphase} (5-25 min):
    \begin{itemize}
      \item Normale Fahrt durch Stadt
      \item Viele Objekte (Fahrzeuge, Fußgänger, Radfahrer)
      \item Häufige Brems- und Beschleunigungsvorgänge
      \item Spurwechsel und Abbiegen
      \item Last: Hoch (viele Objekte, komplexe Szenen)
    \end{itemize}
  
  \item \textbf{Stauphase} (25-35 min):
    \begin{itemize}
      \item Stau auf Autobahn
      \item Sehr viele Objekte (50-100)
      \item Niedrige Geschwindigkeit (0-20 km/h)
      \item Häufige Stopps und Starts
      \item Last: Sehr hoch (extreme Objektdichte)
    \end{itemize}
  
  \item \textbf{Parkphase} (35-40 min):
    \begin{itemize}
      \item Einparken auf Parkplatz
      \item Viele Objekte, enge Räume
      \item Präzise Manöver erforderlich
      \item Alle Sensoren aktiv (Ultraschall, Kameras)
      \item Last: Sehr hoch (viele Objekte, enge Räume)
    \end{itemize}
  
  \item \textbf{Endphase} (40-45 min):
    \begin{itemize}
      \item Fahrzeug abstellen
      \item System-Shutdown
      \item Daten-Speicherung
      \item Last: Niedrig (Shutdown)
    \end{itemize}
\end{enumerate}

\subsubsection{Lastprofile}

Die Lastprofile für jede Phase:

\begin{table}[h]
  \centering
  \caption{Lastprofile: Stadtverkehr-Zyklus}
  \begin{tabular}{lllll}
    \toprule
    Phase & CPU-Last & GPU-Last & Netzwerk-Last & Objektdichte \\
    \midrule
    Start & 20\% & 10\% & 15\% & 0-5 \\
    Fahrt & 60\% & 55\% & 45\% & 20-40 \\
    Stau & 85\% & 75\% & 70\% & 50-100 \\
    Park & 80\% & 70\% & 65\% & 30-60 \\
    Ende & 15\% & 5\% & 10\% & 0-5 \\
    \bottomrule
  \end{tabular}
  \label{tab:stadtverkehr_phasen}
\end{table}

\subsection{Beispiel: Wetter-Szenarien}

Dieses Beispiel zeigt verschiedene Wetter-Szenarien und deren Auswirkungen auf die Architektur.

\subsubsection{Regen-Szenario}

Regen beeinflusst die Perzeption erheblich:

\begin{itemize}
  \item \textbf{Kamera}:
    \begin{itemize}
      \item Reduzierte Sichtweite (50-70\%)
      \item Reflexionen auf der Straße
      \item Wassertropfen auf der Scheibe
      \item Erhöhte Bildverarbeitung erforderlich (Filter, Entrauschung)
      \item CPU-Last: +15-20\%
    \end{itemize}
  
  \item \textbf{Radar}:
    \begin{itemize}
      \item Weniger beeinflusst durch Regen
      \item Leichte Reduzierung der Reichweite
      \item Wichtig für Sensorfusion
    \end{itemize}
  
  \item \textbf{LiDAR}:
    \begin{itemize}
      \item Stärker beeinflusst durch Regen
      \item Wassertropfen reflektieren Laser
      \item Reduzierte Punktwolken-Qualität
      \item Erhöhte Filterung erforderlich
    \end{itemize}
\end{itemize}

\subsubsection{Nebel-Szenario}

Nebel ist besonders herausfordernd:

\begin{itemize}
  \item \textbf{Kamera}:
    \begin{itemize}
      \item Sehr reduzierte Sichtweite (20-40\%)
      \item Kontrastverlust
      \item Erhöhte Bildverarbeitung (Entnebelung, Kontrastverstärkung)
      \item CPU-Last: +25-30\%
    \end{itemize}
  
  \item \textbf{Radar}:
    \begin{itemize}
      \item Weniger beeinflusst
      \item Wird zur primären Perzeptionsquelle
      \item Sensorfusion wird kritisch
    \end{itemize}
  
  \item \textbf{LiDAR}:
    \begin{itemize}
      \item Stark beeinflusst
      \item Nebelpartikel reflektieren Laser
      \item Viele Fehldetektionen
      \item Erhöhte Filterung erforderlich
    \end{itemize}
\end{itemize}

\subsubsection{Schnee-Szenario}

Schnee stellt besondere Herausforderungen:

\begin{itemize}
  \item \textbf{Kamera}:
    \begin{itemize}
      \item Blendung durch Schnee
      \item Kontrastprobleme (weiß auf weiß)
      \item Schneeflocken vor der Kamera
      \item Erhöhte Bildverarbeitung (Entblendung, Kontrastanpassung)
      \item CPU-Last: +20-25\%
    \end{itemize}
  
  \item \textbf{Radar}:
    \begin{itemize}
      \item Schneeflocken können reflektieren
      \item Aber weniger beeinflusst als Kamera/LiDAR
    \end{itemize}
  
  \item \textbf{LiDAR}:
    \begin{itemize}
      \item Stark beeinflusst
      \item Schneeflocken reflektieren Laser
      \item Viele Fehldetektionen
    \end{itemize}
\end{itemize}

\section{Erweiterte Szenario-Definitionen}

Dieser Abschnitt beschreibt erweiterte Methoden zur Definition von Szenarien.

\subsection{Parametrisierte Szenarien}

Szenarien können parametrisiert werden für verschiedene Konfigurationen:

\begin{itemize}
  \item \textbf{Umgebungs-Parameter}: Wetter, Verkehrsdichte, Straßentyp
  \item \textbf{Fahrzeug-Parameter}: Geschwindigkeit, Beschleunigung, Last
  \item \textbf{System-Parameter}: CPU-Frequenz, Netzwerk-Bandbreite, Sensor-Framerate
  \item \textbf{Fehler-Parameter}: Fehlerrate, Fehlertyp, Fehlerzeitpunkt
\end{itemize}

\subsection{Szenario-Kombinationen}

Komplexe Szenarien können aus einfachen Szenarien kombiniert werden:

\begin{itemize}
  \item \textbf{Sequenzielle Kombination}: Mehrere Szenarien nacheinander
  \item \textbf{Parallele Kombination}: Mehrere Szenarien gleichzeitig
  \item \textbf{Bedingte Kombination}: Szenarien basierend auf Bedingungen
  \item \textbf{Probabilistische Kombination}: Zufällige Kombinationen
\end{itemize}

\subsection{Szenario-Templates}

Wiederverwendbare Szenario-Templates:

\begin{verbatim}
{
  "scenario_template": {
    "name": "urban_driving",
    "parameters": {
      "traffic_density": "high|medium|low",
      "weather": "sunny|rain|fog|snow",
      "speed": "0-50 km/h"
    },
    "phases": [
      {
        "name": "start",
        "duration": "5 min",
        "load": "low"
      },
      {
        "name": "driving",
        "duration": "20 min",
        "load": "high"
      }
    ]
  }
}
\end{verbatim}

\section{Zusammenfassung}

Dieses Kapitel hat verschiedene Szenarien und Use-Cases für die Simulation beschrieben:

\begin{itemize}
  \item \textbf{Nominal-Szenarien}: Standard-Betriebsbedingungen
  \item \textbf{Stress-Szenarien}: Stau, Wetter, extreme Last
  \item \textbf{Fehlerszenarien}: ECU-/Link-Ausfälle, Queue-Überlauf, Clock-Drift
  \item \textbf{Design-Varianten}: Redundanz, Bandbreite, Scheduling
  \item \textbf{Use-Case-spezifische Szenarien}: Autonomes Fahren, Fahrerassistenz, Fracht-Management
\end{itemize}

Diese Szenarien ermöglichen eine umfassende Bewertung der Architektur unter verschiedenen Bedingungen und helfen, Schwachstellen und Optimierungspotenziale zu identifizieren.
