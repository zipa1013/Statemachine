\chapter{Energiemanagement in E/E-Architekturen}\label{chap:energiemanagement}

\noindent
Dieses Kapitel beschreibt die Modellierung und Optimierung des Energieverbrauchs in E/E-Architekturen. Energiemanagement ist besonders wichtig für elektrische Fahrzeuge, wo der Energieverbrauch direkt die Reichweite beeinflusst.

\section{Energieverbrauch in E/E-Architekturen}

Der Energieverbrauch in E/E-Architekturen setzt sich zusammen aus:

\begin{itemize}
  \item \textbf{Rechenknoten}: CPU, GPU, NPU, Memory
  \item \textbf{Kommunikation}: Netzwerk-Interfaces, Switches
  \item \textbf{Sensoren}: Kameras, Radar, LiDAR, etc.
  \item \textbf{Aktoren}: EPS, EHB, E-Motor, etc.
  \item \textbf{Auxiliary}: Beleuchtung, Klima, etc.
\end{itemize}

\subsection{Power-States}

Rechenknoten können verschiedene Power-States haben:

\begin{itemize}
  \item \textbf{Active}: Volle Leistung, alle Funktionen aktiv
  \item \textbf{Idle}: Niedrige Leistung, keine aktiven Tasks
  \item \textbf{Sleep}: Sehr niedrige Leistung, minimale Funktionen
  \item \textbf{Off}: Keine Energieversorgung
\end{itemize}

\section{Energie-Optimierung}

Energie-Optimierung kann auf verschiedenen Ebenen erfolgen:

\subsection{Hardware-Optimierung}

\begin{itemize}
  \item \textbf{Dynamic Voltage and Frequency Scaling (DVFS)}: Anpassung von Spannung und Frequenz
  \item \textbf{Power-Gating}: Abschalten nicht genutzter Komponenten
  \item \textbf{Clock-Gating}: Abschalten nicht genutzter Clock-Domains
  \item \textbf{Low-Power-Modi}: Nutzung von Low-Power-Modi
\end{itemize}

\subsection{Software-Optimierung}

\begin{itemize}
  \item \textbf{Task-Scheduling}: Energie-effizientes Scheduling
  \item \textbf{Load-Balancing}: Gleichmäßige Lastverteilung
  \item \textbf{Adaptive Quality}: Anpassung der Qualität basierend auf verfügbarer Energie
  \item \textbf{Predictive Shutdown}: Vorhersagbares Abschalten nicht benötigter Komponenten
\end{itemize}

\section{Modellierung}

Der Energieverbrauch wird in der Simulation modelliert:

\begin{equation}
E_{total} = \sum_{i=1}^{N} \left( P_i \times t_i \right)
\end{equation}

wobei:
\begin{itemize}
  \item $P_i$: Leistung in Zustand $i$
  \item $t_i$: Zeit in Zustand $i$
  \item $N$: Anzahl der Zustände
\end{itemize}

\section{Zusammenfassung}

Dieses Kapitel hat die Modellierung und Optimierung des Energieverbrauchs in E/E-Architekturen beschrieben. Energiemanagement ist entscheidend für die Effizienz elektrischer Fahrzeuge und muss bereits in der Architektur-Phase berücksichtigt werden.

\section{Erweiterte Energie-Modellierung}

Dieser Abschnitt beschreibt erweiterte Methoden zur Modellierung des Energieverbrauchs.

\subsection{Detaillierte Power-State-Modellierung}

Rechenknoten haben verschiedene Power-States mit unterschiedlichen Leistungen:

\begin{table}[h]
  \centering
  \caption{Power-States für Central Compute}
  \begin{tabular}{lllll}
    \toprule
    State & CPU & GPU & NPU & Gesamt \\
    \midrule
    Active & 25W & 30W & 15W & 70W \\
    Idle & 5W & 2W & 1W & 8W \\
    Sleep & 0.5W & 0.1W & 0.1W & 0.7W \\
    Off & 0W & 0W & 0W & 0W \\
    \bottomrule
  \end{tabular}
  \label{tab:power_states_cc}
\end{table}

\subsection{DVFS-Modellierung}

Dynamic Voltage and Frequency Scaling (DVFS) ermöglicht Energie-Optimierung:

\begin{equation}
P_{DVFS}(f, V) = C_{eff} \times V^2 \times f + P_{static}
\end{equation}

wobei:
\begin{itemize}
  \item $f$: Frequenz
  \item $V$: Spannung
  \item $C_{eff}$: Effektive Kapazität
  \item $P_{static}$: Statische Leistung
\end{itemize}

Die optimale Frequenz für minimale Energie bei gegebener Last:

\begin{equation}
f_{opt} = \sqrt{\frac{L}{C_{eff} \times V^2}}
\end{equation}

\subsection{Temperaturabhängigkeit}

Der Energieverbrauch hängt von der Temperatur ab:

\begin{equation}
P_{total}(T) = P_{dyn} + P_{leak}(T) + P_{static}
\end{equation}

wobei der Leckstrom temperaturabhängig ist:

\begin{equation}
I_{leak}(T) = I_0 \times e^{\frac{E_a}{k_B T}}
\end{equation}

\section{Erweiterte Energie-Optimierungs-Strategien}

Dieser Abschnitt beschreibt erweiterte Strategien zur Energie-Optimierung.

\subsection{Task-Scheduling für Energie-Effizienz}

Energie-effizientes Task-Scheduling:

\begin{itemize}
  \item \textbf{Consolidation}: Zusammenfassung von Tasks auf wenige Cores
  \item \textbf{Migration}: Migration von Tasks zwischen Cores
  \item \textbf{Frequency-Scaling}: Anpassung der Frequenz basierend auf Last
  \item \textbf{Power-Gating}: Abschalten nicht genutzter Cores
\end{itemize}

\subsection{Adaptive-Quality}

Anpassung der Qualität basierend auf verfügbarer Energie:

\begin{table}[h]
  \centering
  \caption{Adaptive-Quality-Strategien}
  \begin{tabular}{lllll}
    \toprule
    Energie-Level & Qualität & Latenz & Genauigkeit & Energie \\
    \midrule
    Hoch & Hoch & Niedrig & Hoch & Hoch \\
    Mittel & Mittel & Mittel & Mittel & Mittel \\
    Niedrig & Niedrig & Hoch & Niedrig & Niedrig \\
    \bottomrule
  \end{tabular}
  \label{tab:adaptive_quality}
\end{table}

\subsection{Predictive-Power-Management}

Vorhersagbares Power-Management:

\begin{itemize}
  \item \textbf{Load-Prediction}: Vorhersage der zukünftigen Last
  \item \textbf{Preemptive-Scaling}: Proaktive Skalierung
  \item \textbf{Energy-Budgeting}: Energie-Budget-Verwaltung
  \item \textbf{Optimization}: Optimierung basierend auf Vorhersagen
\end{itemize}

\section{Erweiterte Energie-Analyse}

Dieser Abschnitt beschreibt erweiterte Methoden zur Analyse des Energieverbrauchs.

\subsection{Energie-Profil-Analyse}

Energie-Profile für verschiedene Szenarien:

\begin{table}[h]
  \centering
  \caption{Energie-Profile für verschiedene Szenarien}
  \begin{tabular}{llll}
    \toprule
    Szenario & Energie (Wh/km) & Komponente & Anteil \\
    \midrule
    Stadtverkehr & 25 & E-Motor & 60\% \\
    Stadtverkehr & 25 & E/E-System & 15\% \\
    Stadtverkehr & 25 & Klima & 20\% \\
    Stadtverkehr & 25 & Sonstiges & 5\% \\
    Autobahn & 30 & E-Motor & 70\% \\
    Autobahn & 30 & E/E-System & 10\% \\
    Autobahn & 30 & Klima & 15\% \\
    Autobahn & 30 & Sonstiges & 5\% \\
    \bottomrule
  \end{tabular}
  \label{tab:energy_profiles}
\end{table}

\subsection{Energie-Optimierungs-Potenzial}

Potenzial für Energie-Optimierung:

\begin{itemize}
  \item \textbf{Hardware}: 10-20\% durch effizientere Hardware
  \item \textbf{Software}: 15-25\% durch optimierte Software
  \item \textbf{Architektur}: 20-30\% durch optimierte Architektur
  \item \textbf{Gesamt}: 30-50\% durch kombinierte Optimierungen
\end{itemize}

\section{Erweiterte Energie-Simulation}

Dieser Abschnitt beschreibt erweiterte Methoden zur Simulation des Energieverbrauchs.

\subsection{Multi-Level-Energie-Simulation}

Energie-Simulation auf verschiedenen Ebenen:

\begin{itemize}
  \item \textbf{System-Level}: Gesamtsystem-Energieverbrauch
  \item \textbf{Component-Level}: Komponenten-Energieverbrauch
  \item \textbf{Task-Level}: Task-Energieverbrauch
  \item \textbf{Instruction-Level}: Instruction-Energieverbrauch
\end{itemize}

\subsection{Energie-Validierung}

Validierung der Energie-Simulation:

\begin{itemize}
  \item \textbf{Benchmarking}: Vergleich mit realen Messungen
  \item \textbf{Sensitivity-Analysis}: Analyse der Sensitivität
  \item \textbf{Uncertainty-Quantification}: Quantifizierung der Unsicherheit
\end{itemize}

