\chapter{Standards und Compliance}\label{chap:standards}

\noindent
Dieses Kapitel beschreibt die relevanten Standards und Compliance-Anforderungen für E/E-Architekturen und wie diese in der Modellierung und Simulation berücksichtigt werden.

\section{ISO 26262 - Funktionale Sicherheit}

ISO 26262 ist der Standard für funktionale Sicherheit in Kraftfahrzeugen und definiert Anforderungen für die Entwicklung sicherheitskritischer Systeme.

\subsection{ASIL-Level}

ASIL (Automotive Safety Integrity Level) definiert vier Sicherheitsstufen:

\begin{itemize}
  \item \textbf{ASIL A}: Niedrigste Sicherheitsstufe, geringe Anforderungen
  \item \textbf{ASIL B}: Mittlere Sicherheitsstufe, moderate Anforderungen
  \item \textbf{ASIL C}: Hohe Sicherheitsstufe, hohe Anforderungen
  \item \textbf{ASIL D}: Höchste Sicherheitsstufe, höchste Anforderungen
\end{itemize}

\subsection{Modellierung in PREEvision}

ASIL-Level werden in PREEvision wie folgt modelliert:

\begin{itemize}
  \item \textbf{ECU-ASIL}: ASIL-Level für ECUs
  \item \textbf{SWC-ASIL}: ASIL-Level für Software-Komponenten
  \item \textbf{Signal-ASIL}: ASIL-Level für Signale
  \item \textbf{Partitionierung}: ASIL-Isolation durch Partitionierung
\end{itemize}

\subsection{Simulation}

Die Simulation berücksichtigt ASIL-Anforderungen:

\begin{itemize}
  \item \textbf{Verfügbarkeit}: ASIL-spezifische Verfügbarkeitsanforderungen
  \item \textbf{Redundanz}: ASIL-spezifische Redundanzanforderungen
  \item \textbf{Fehlerbehandlung}: ASIL-spezifische Fehlerbehandlung
\end{itemize}

\section{AUTOSAR}

AUTOSAR (AUTomotive Open System ARchitecture) ist ein Standard für Automotive-Software-Architekturen.

\subsection{AUTOSAR Classic}

AUTOSAR Classic für eingebettete Systeme:

\begin{itemize}
  \item \textbf{SWCs}: Software Components mit Ports und Interfaces
  \item \textbf{RTE}: Runtime Environment für Kommunikation
  \item \textbf{BSW}: Basic Software Layer
  \item \textbf{ECU Configuration}: ECU-spezifische Konfiguration
\end{itemize}

\subsection{AUTOSAR Adaptive}

AUTOSAR Adaptive für hochperformante Systeme:

\begin{itemize}
  \item \textbf{Adaptive Applications}: C++-basierte Anwendungen
  \item \textbf{ARA}: Adaptive AUTOSAR Runtime
  \item \textbf{Service-Orientation}: SOME/IP, DDS
  \item \textbf{Execution Management}: Lifecycle-Management
\end{itemize}

\section{TSN-Standards}

Time-Sensitive Networking (TSN) Standards für deterministische Kommunikation:

\subsection{IEEE 802.1 Standards}

\begin{itemize}
  \item \textbf{IEEE 802.1AS}: gPTP (generalized Precision Time Protocol)
  \item \textbf{IEEE 802.1Qbv}: Time-Aware Shaping
  \item \textbf{IEEE 802.1Qbu}: Frame Preemption
  \item \textbf{IEEE 802.1Qcc}: TSN Configuration
\end{itemize}

\subsection{Modellierung}

TSN-Standards werden in PREEvision modelliert:

\begin{itemize}
  \item \textbf{Gate-Schedules}: Konfiguration von Gate-Schedules
  \item \textbf{Time-Synchronisation}: gPTP-Konfiguration
  \item \textbf{Traffic-Shaping}: Traffic-Shaping-Parameter
  \item \textbf{Frame-Preemption}: Preemption-Konfiguration
\end{itemize}

\section{UN R155 - Cybersecurity}

UN Regulation 155 definiert Anforderungen für Cybersecurity in Fahrzeugen:

\begin{itemize}
  \item \textbf{CSMS}: Cybersecurity Management System
  \item \textbf{Threat-Analysis}: Bedrohungsanalyse
  \item \textbf{Risk-Assessment}: Risikobewertung
  \item \textbf{Security-by-Design}: Sicherheit durch Design
\end{itemize}

\section{Erweiterte Standards-Details}

Dieser Abschnitt beschreibt erweiterte Details zu relevanten Standards.

\subsection{ISO 21434 - Cybersecurity}

ISO 21434 definiert Anforderungen für Cybersecurity in Fahrzeugen:

\begin{itemize}
  \item \textbf{CSMS}: Cybersecurity Management System
  \item \textbf{Threat-Analysis}: Systematische Bedrohungsanalyse
  \item \textbf{Risk-Assessment}: Risikobewertung und -behandlung
  \item \textbf{Security-by-Design}: Sicherheit durch Design
  \item \textbf{Incident-Response}: Reaktion auf Sicherheitsvorfälle
\end{itemize}

\subsection{UN R156 - Software-Updates}

UN Regulation 156 definiert Anforderungen für Software-Updates:

\begin{itemize}
  \item \textbf{SUMS}: Software Update Management System
  \item \textbf{Update-Verifizierung}: Verifizierung von Updates
  \item \textbf{Rollback-Mechanismen}: Mechanismen zum Zurücksetzen
  \item \textbf{Update-Dokumentation}: Dokumentation von Updates
\end{itemize}

\subsection{ISO 14229 - UDS}

ISO 14229 definiert Unified Diagnostic Services (UDS):

\begin{itemize}
  \item \textbf{Diagnostic-Services}: Verschiedene Diagnose-Services
  \item \textbf{Session-Control}: Kontrolle von Diagnose-Sessions
  \item \textbf{Data-Transfer}: Übertragung von Diagnose-Daten
  \item \textbf{Security-Access}: Zugriffskontrolle für Diagnose
\end{itemize}

\section{Zusammenfassung}

Dieses Kapitel hat relevante Standards und Compliance-Anforderungen beschrieben, die bei der Entwicklung von E/E-Architekturen berücksichtigt werden müssen. Die Modellierung und Simulation müssen diese Anforderungen unterstützen, um sicherzustellen, dass die Architektur konform ist.

