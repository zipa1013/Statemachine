\chapter{Cybersecurity in E/E-Architekturen}\label{chap:cybersecurity}

\noindent
Dieses Kapitel beschreibt die Modellierung und Simulation von Cybersecurity-Aspekten in E/E-Architekturen. Cybersecurity wird zunehmend wichtig in vernetzten Fahrzeugen und muss bereits in der Architektur-Phase berücksichtigt werden.

\section{Cybersecurity-Anforderungen}

Moderne E/E-Architekturen müssen verschiedene Cybersecurity-Anforderungen erfüllen:

\begin{itemize}
  \item \textbf{Vertraulichkeit}: Schutz vor unberechtigtem Zugriff auf Daten
  \item \textbf{Integrität}: Schutz vor Manipulation von Daten
  \item \textbf{Verfügbarkeit}: Schutz vor Denial-of-Service-Angriffen
  \item \textbf{Authentifizierung}: Verifizierung der Identität von Kommunikationspartnern
  \item \textbf{Autorisierung}: Kontrolle des Zugriffs auf Ressourcen
  \item \textbf{Non-Repudiation}: Nachweisbarkeit von Aktionen
\end{itemize}

\section{Modellierung von Cybersecurity}

Cybersecurity-Aspekte werden in PREEvision wie folgt modelliert:

\subsection{Sicherheitszonen}

Sicherheitszonen definieren Bereiche mit unterschiedlichen Sicherheitsanforderungen:

\begin{itemize}
  \item \textbf{Trusted Zone}: Höchste Sicherheit (z.\,B. sicherheitskritische Funktionen)
  \item \textbf{Secure Zone}: Hohe Sicherheit (z.\,B. Infotainment)
  \item \textbf{Untrusted Zone}: Niedrige Sicherheit (z.\,B. externe Schnittstellen)
\end{itemize}

\subsection{Sicherheitsmechanismen}

Verschiedene Sicherheitsmechanismen werden modelliert:

\begin{itemize}
  \item \textbf{Verschlüsselung}: AES, RSA, ECC
  \item \textbf{Digitale Signaturen}: RSA, ECDSA
  \item \textbf{Hash-Funktionen}: SHA-256, SHA-3
  \item \textbf{Secure Boot}: Verifizierung beim Start
  \item \textbf{Secure Update}: Sichere OTA-Updates
  \item \textbf{Intrusion Detection}: Erkennung von Angriffen
\end{itemize}

\section{Simulation von Cybersecurity}

Die Simulation von Cybersecurity-Aspekten ermöglicht die Bewertung von Sicherheitsmaßnahmen:

\subsection{Angriffs-Szenarien}

Verschiedene Angriffs-Szenarien werden simuliert:

\begin{itemize}
  \item \textbf{Man-in-the-Middle}: Abfangen und Manipulation von Kommunikation
  \item \textbf{Denial-of-Service}: Überlastung von Systemen
  \item \textbf{Replay-Angriffe}: Wiederverwendung von Nachrichten
  \item \textbf{Privilege Escalation}: Erhöhung von Berechtigungen
\end{itemize}

\subsection{Sicherheits-Performance}

Die Performance von Sicherheitsmechanismen wird modelliert:

\begin{table}[h]
  \centering
  \caption{Performance von Sicherheitsmechanismen}
  \begin{tabular}{llll}
    \toprule
    Mechanismus & Latenz & CPU-Last & Anwendung \\
    \midrule
    AES-128 (SW) & 0.5 ms & 5\% & Datenverschlüsselung \\
    AES-128 (HW) & 0.1 ms & 1\% & Datenverschlüsselung \\
    RSA-2048 & 10 ms & 15\% & Signatur-Verifizierung \\
    ECDSA-256 & 2 ms & 8\% & Signatur-Verifizierung \\
    SHA-256 & 0.2 ms & 2\% & Hash-Berechnung \\
    \bottomrule
  \end{tabular}
  \label{tab:security_performance}
\end{table}

\section{Zusammenfassung}

Dieses Kapitel hat die Modellierung und Simulation von Cybersecurity-Aspekten in E/E-Architekturen beschrieben. Die wichtigsten Aspekte umfassen:

\begin{itemize}
  \item \textbf{Anforderungen}: Vertraulichkeit, Integrität, Verfügbarkeit, etc.
  \item \textbf{Modellierung}: Sicherheitszonen, Sicherheitsmechanismen
  \item \textbf{Simulation}: Angriffs-Szenarien, Sicherheits-Performance
\end{itemize}

Cybersecurity muss bereits in der Architektur-Phase berücksichtigt werden, um sicherzustellen, dass die Architektur sicher ist.

\section{Erweiterte Cybersecurity-Modellierung}

Dieser Abschnitt beschreibt erweiterte Aspekte der Cybersecurity-Modellierung.

\subsection{Threat-Modellierung}

Threat-Modellierung identifiziert potenzielle Bedrohungen:

\begin{itemize}
  \item \textbf{STRIDE}: Spoofing, Tampering, Repudiation, Information Disclosure, Denial of Service, Elevation of Privilege
  \item \textbf{Attack-Trees}: Hierarchische Darstellung von Angriffen
  \item \textbf{Attack-Surfaces}: Identifikation von Angriffsflächen
  \item \textbf{Risk-Assessment}: Bewertung von Risiken
\end{itemize}

\subsection{Security-by-Design}

Security-by-Design integriert Sicherheit von Anfang an:

\begin{itemize}
  \item \textbf{Secure-Architecture}: Architektur mit Sicherheit im Fokus
  \item \textbf{Defense-in-Depth}: Mehrschichtige Sicherheitsmaßnahmen
  \item \textbf{Least-Privilege}: Minimale Berechtigungen
  \item \textbf{Secure-Communication}: Verschlüsselte Kommunikation
\end{itemize}

\section{Erweiterte Cybersecurity-Mechanismen}

Dieser Abschnitt beschreibt erweiterte Cybersecurity-Mechanismen.

\subsection{Secure-Boot}

Secure-Boot verifiziert die Integrität beim Start:

\begin{enumerate}
  \item \textbf{ROM-Code}: Verifiziert Bootloader-Signatur
  \item \textbf{Bootloader}: Verifiziert OS-Signatur
  \item \textbf{OS}: Verifiziert Anwendungs-Signaturen
  \item \textbf{Chain-of-Trust}: Vertrauenskette von Hardware zu Software
\end{enumerate}

\subsection{Secure-Update}

Secure-Update ermöglicht sichere OTA-Updates:

\begin{itemize}
  \item \textbf{Signatur-Verifizierung}: Verifizierung der Update-Signatur
  \item \textbf{Rollback-Mechanismus}: Zurücksetzen bei Problemen
  \item \textbf{Incremental-Updates}: Nur Änderungen übertragen
  \item \textbf{Encrypted-Transport}: Verschlüsselte Übertragung
\end{itemize}

\subsection{Intrusion-Detection-System (IDS)}

IDS erkennt Angriffe in Echtzeit:

\begin{itemize}
  \item \textbf{Signature-Based}: Erkennung bekannter Angriffsmuster
  \item \textbf{Anomaly-Based}: Erkennung von Anomalien
  \item \textbf{Behavior-Based}: Erkennung von Verhaltensänderungen
  \item \textbf{Machine-Learning}: ML-basierte Erkennung
\end{itemize}

\section{Erweiterte Cybersecurity-Simulation}

Dieser Abschnitt beschreibt erweiterte Methoden zur Simulation von Cybersecurity.

\subsection{Attack-Scenario-Modellierung}

Verschiedene Angriffs-Szenarien werden modelliert:

\begin{table}[h]
  \centering
  \caption{Angriffs-Szenarien}
  \begin{tabular}{llll}
    \toprule
    Szenario & Typ & Impact & Detection \\
    \midrule
    Man-in-the-Middle & Passiv & Hoch & Schwierig \\
    Denial-of-Service & Aktiv & Mittel & Einfach \\
    Replay-Angriff & Aktiv & Mittel & Mittel \\
    Privilege Escalation & Aktiv & Sehr hoch & Schwierig \\
    \bottomrule
  \end{tabular}
  \label{tab:attack_scenarios}
\end{table}

\subsection{Security-Performance-Analyse}

Die Performance von Sicherheitsmechanismen wird analysiert:

\begin{equation}
L_{security} = L_{encryption} + L_{signature} + L_{verification} + L_{overhead}
\end{equation}

wobei:
\begin{itemize}
  \item $L_{encryption}$: Verschlüsselungs-Latenz
  \item $L_{signature}$: Signatur-Latenz
  \item $L_{verification}$: Verifizierungs-Latenz
  \item $L_{overhead}$: Overhead-Latenz
\end{itemize}

\subsection{Security-vs-Performance-Trade-off}

Es gibt einen Trade-off zwischen Sicherheit und Performance:

\begin{itemize}
  \item \textbf{Höhere Sicherheit}: Mehr Verschlüsselung, höhere Latenz
  \item \textbf{Niedrigere Latenz}: Weniger Verschlüsselung, niedrigere Sicherheit
  \item \textbf{Optimierung}: Hardware-Beschleunigung, selektive Verschlüsselung
\end{itemize}

\section{Erweiterte Cybersecurity-Standards}

Dieser Abschnitt beschreibt relevante Cybersecurity-Standards.

\subsection{ISO 21434}

ISO 21434 definiert Anforderungen für Cybersecurity:

\begin{itemize}
  \item \textbf{CSMS}: Cybersecurity Management System
  \item \textbf{Threat-Analysis}: Systematische Bedrohungsanalyse
  \item \textbf{Risk-Assessment}: Risikobewertung und -behandlung
  \item \textbf{Security-by-Design}: Sicherheit durch Design
\end{itemize}

\subsection{UN R155}

UN Regulation 155 definiert Anforderungen für Cybersecurity:

\begin{itemize}
  \item \textbf{CSMS}: Cybersecurity Management System
  \item \textbf{Type-Approval}: Typgenehmigung für Cybersecurity
  \item \textbf{Incident-Response}: Reaktion auf Sicherheitsvorfälle
\end{itemize}

\subsection{SAE J3061}

SAE J3061 definiert Guidelines für Cybersecurity:

\begin{itemize}
  \item \textbf{Cybersecurity-Process}: Prozess für Cybersecurity
  \item \textbf{Threat-Analysis}: Bedrohungsanalyse
  \item \textbf{Risk-Assessment}: Risikobewertung
  \item \textbf{Security-Testing}: Sicherheitstests
\end{itemize}

\section{Erweiterte Cybersecurity-Implementierung}

Dieser Abschnitt beschreibt erweiterte Implementierungs-Aspekte der Cybersecurity.

\subsection{HSM (Hardware Security Module)}

HSM bietet hardware-basierte Sicherheit:

\begin{itemize}
  \item \textbf{Key-Management}: Sichere Schlüssel-Verwaltung
  \item \textbf{Crypto-Acceleration}: Hardware-Beschleunigung für Kryptografie
  \item \textbf{Secure-Storage}: Sichere Speicherung von Schlüsseln
  \item \textbf{Tamper-Resistance}: Manipulations-Schutz
\end{itemize}

\subsection{Secure Communication}

Sichere Kommunikation zwischen Komponenten:

\begin{itemize}
  \item \textbf{TLS/DTLS}: Transport Layer Security für Ethernet
  \item \textbf{SecOC}: Secure Onboard Communication für CAN
  \item \textbf{IPsec}: IP Security für IP-basierte Kommunikation
  \item \textbf{MAC-Sec}: MAC Security für Ethernet
\end{itemize}

\subsection{Secure Gateway}

Secure Gateway trennt sicherheitskritische Bereiche:

\begin{itemize}
  \item \textbf{Firewall}: Paket-Filterung
  \item \textbf{Intrusion-Detection}: Erkennung von Angriffen
  \item \textbf{VPN}: Virtuelle Private Networks
  \item \textbf{Access-Control}: Zugriffskontrolle
\end{itemize}

\section{Erweiterte Cybersecurity-Simulation-Beispiele}

Dieser Abschnitt beschreibt erweiterte Beispiele für Cybersecurity-Simulation.

\subsection{Beispiel: Man-in-the-Middle-Angriff}

Simulation eines Man-in-the-Middle-Angriffs:

\begin{itemize}
  \item \textbf{Szenario}: Angreifer fängt Kommunikation ab
  \item \textbf{Impact}: Manipulation von Daten, Verlust von Vertraulichkeit
  \item \textbf{Detection}: Erkennung durch Anomalie-Erkennung
  \item \textbf{Mitigation}: Verschlüsselung, Authentifizierung
\end{itemize}

\subsection{Beispiel: Denial-of-Service-Angriff}

Simulation eines Denial-of-Service-Angriffs:

\begin{itemize}
  \item \textbf{Szenario}: Angreifer überlastet System mit Anfragen
  \item \textbf{Impact}: System-Unverfügbarkeit, Performance-Degradation
  \item \textbf{Detection}: Erkennung durch Last-Monitoring
  \item \textbf{Mitigation}: Rate-Limiting, Traffic-Filtering
\end{itemize}

\subsection{Beispiel: Privilege-Escalation}

Simulation einer Privilege-Escalation:

\begin{itemize}
  \item \textbf{Szenario}: Angreifer erhöht Berechtigungen
  \item \textbf{Impact}: Unautorisierter Zugriff, System-Kompromittierung
  \item \textbf{Detection}: Erkennung durch Access-Control-Monitoring
  \item \textbf{Mitigation}: Least-Privilege-Prinzip, Regular-Audits
\end{itemize}

\section{Erweiterte Cybersecurity-Metriken}

Dieser Abschnitt beschreibt erweiterte Metriken für Cybersecurity.

\subsection{Security-Metriken}

Verschiedene Security-Metriken werden verwendet:

\begin{itemize}
  \item \textbf{Mean-Time-to-Detect (MTTD)}: Durchschnittliche Zeit bis zur Erkennung
  \item \textbf{Mean-Time-to-Respond (MTTR)}: Durchschnittliche Zeit bis zur Reaktion
  \item \textbf{Attack-Success-Rate}: Erfolgsrate von Angriffen
  \item \textbf{False-Positive-Rate}: Rate von Fehlalarmen
\end{itemize}

\subsection{Security-Performance-Trade-off}

Trade-off zwischen Sicherheit und Performance:

\begin{equation}
P_{secure} = P_{baseline} + P_{encryption} + P_{signature} + P_{verification} + P_{overhead}
\end{equation}

wobei:
\begin{itemize}
  \item $P_{secure}$: Performance mit Sicherheit
  \item $P_{baseline}$: Baseline-Performance
  \item $P_{encryption}$: Performance-Overhead durch Verschlüsselung
  \item $P_{signature}$: Performance-Overhead durch Signaturen
  \item $P_{verification}$: Performance-Overhead durch Verifizierung
  \item $P_{overhead}$: Allgemeiner Overhead
\end{itemize}

\section{Erweiterte Cybersecurity-Best-Practices}

Dieser Abschnitt beschreibt Best Practices für Cybersecurity.

\subsection{Defense-in-Depth}

Mehrschichtige Sicherheitsmaßnahmen:

\begin{enumerate}
  \item \textbf{Network-Security}: Firewalls, IDS, VPN
  \item \textbf{Application-Security}: Secure Coding, Input-Validation
  \item \textbf{Data-Security}: Encryption, Access-Control
  \item \textbf{Device-Security}: Secure Boot, HSM, Tamper-Resistance
\end{enumerate}

\subsection{Security-Monitoring}

Kontinuierliche Überwachung der Sicherheit:

\begin{itemize}
  \item \textbf{Log-Analysis}: Analyse von Log-Dateien
  \item \textbf{Anomaly-Detection}: Erkennung von Anomalien
  \item \textbf{Threat-Intelligence}: Bedrohungs-Intelligence
  \item \textbf{Incident-Response}: Reaktion auf Vorfälle
\end{itemize}

\subsection{Security-Testing}

Regelmäßige Sicherheitstests:

\begin{itemize}
  \item \textbf{Penetration-Testing}: Penetrationstests
  \item \textbf{Vulnerability-Scanning}: Schwachstellen-Scans
  \item \textbf{Code-Review}: Code-Reviews auf Sicherheitsprobleme
  \item \textbf{Fuzz-Testing}: Fuzz-Tests für Eingabe-Validierung
\end{itemize}

