\chapter{Risiken und Gegenmaßnahmen}\label{chap:risiken}

\noindent
Dieses Kapitel beschreibt die identifizierten Risiken des Projekts und die geplanten Gegenmaßnahmen. Die Risiken umfassen technische Herausforderungen, Datenverfügbarkeit, Komplexität und Reproduzierbarkeit.

\section{Technische Risiken}

\subsection{Toolkopplung und Exportformate}

\subsubsection{Risiko}

\begin{itemize}
  \item \textbf{Beschreibung}: PREEvision-Exportformate könnten sich ändern oder unvollständig sein
  \item \textbf{Wahrscheinlichkeit}: Mittel
  \item \textbf{Auswirkung}: Hoch (kann Projekt verzögern)
  \item \textbf{Erkennung}: Frühe Tests der Export-Funktionalität
\end{itemize}

\subsubsection{Gegenmaßnahmen}

\begin{itemize}
  \item \textbf{IM-Schicht}: Intermediate Model als Abstraktionsebene, unabhängig von Export-Format
  \item \textbf{Adapter-Pattern}: Flexible Parser für verschiedene Formate
  \item \textbf{Frühe Validierung}: Tests der Export-Funktionalität in Phase 2
  \item \textbf{Alternative Exporte}: Nutzung mehrerer Exportformate (REST, CSV, XML)
  \item \textbf{Kontakt zu PREEvision-Support}: Regelmäßiger Kontakt für Format-Fragen
\end{itemize}

\subsection{Simulationsplattform-Kompatibilität}

\subsubsection{Risiko}

\begin{itemize}
  \item \textbf{Beschreibung}: Zielplattform könnte nicht alle erforderlichen Features unterstützen
  \item \textbf{Wahrscheinlichkeit}: Mittel
  \item \textbf{Auswirkung}: Mittel (kann Anpassungen erfordern)
\end{itemize}

\subsubsection{Gegenmaßnahmen}

\begin{itemize}
  \item \textbf{Frühe Evaluation}: Evaluation der Plattform in Phase 1
  \item \textbf{Multi-Plattform-Ansatz}: Unterstützung mehrerer Plattformen
  \item \textbf{Feature-Mapping}: Detaillierte Analyse der Feature-Kompatibilität
  \item \textbf{Workarounds}: Entwicklung von Workarounds für fehlende Features
\end{itemize}

\section{Datenverfügbarkeit}

\subsection{Fehlende oder unvollständige Daten}

\subsubsection{Risiko}

\begin{itemize}
  \item \textbf{Beschreibung}: WCET, Datenraten oder andere Parameter könnten fehlen
  \item \textbf{Wahrscheinlichkeit}: Hoch
  \item \textbf{Auswirkung}: Mittel (kann Genauigkeit beeinträchtigen)
\end{itemize}

\subsubsection{Gegenmaßnahmen}

\begin{itemize}
  \item \textbf{Default-Werte}: Definition von realistischen Default-Werten
  \item \textbf{Schätzungen}: Entwicklung von Schätzungs-Methoden basierend auf ähnlichen Komponenten
  \item \textbf{Benchmarks}: Nutzung von Benchmark-Daten für typische Komponenten
  \item \textbf{Dokumentation}: Klare Dokumentation fehlender Daten
  \item \textbf{Sensitivitäts-Analyse}: Analyse der Auswirkung fehlender Daten
\end{itemize}

\subsection{Datenqualität}

\subsubsection{Risiko}

\begin{itemize}
  \item \textbf{Beschreibung}: Exportierte Daten könnten inkonsistent oder fehlerhaft sein
  \item \textbf{Wahrscheinlichkeit}: Mittel
  \item \textbf{Auswirkung}: Mittel (kann zu falschen Ergebnissen führen)
\end{itemize}

\subsubsection{Gegenmaßnahmen}

\begin{itemize}
  \item \textbf{Validierung}: Umfassende Validierung der Eingabedaten
  \item \textbf{Konsistenz-Checks}: Automatische Prüfung auf Konsistenz
  \item \textbf{Fehlerbehandlung}: Robuste Fehlerbehandlung und -meldungen
  \item \textbf{Manuelle Prüfung}: Manuelle Prüfung kritischer Daten
\end{itemize}

\section{Komplexität}

\subsection{Architektur-Komplexität}

\subsubsection{Risiko}

\begin{itemize}
  \item \textbf{Beschreibung}: Reale Architekturen könnten zu komplex für die Transformation sein
  \item \textbf{Wahrscheinlichkeit}: Mittel
  \item \textbf{Auswirkung}: Hoch (kann Transformation unmöglich machen)
\end{itemize}

\subsubsection{Gegenmaßnahmen}

\begin{itemize}
  \item \textbf{Inkrementelle Schritte}: Schrittweise Erhöhung der Komplexität
  \item \textbf{Abstraktion}: Abstraktion komplexer Aspekte im IM
  \item \textbf{Modularität}: Modulare Transformation für verschiedene Aspekte
  \item \textbf{Beispiel-Architekturen}: Beginn mit einfachen Beispielen
  \item \textbf{Skalierbarkeit}: Design für Skalierbarkeit von Anfang an
\end{itemize}

\subsection{Metrik-Komplexität}

\subsubsection{Risiko}

\begin{itemize}
  \item \textbf{Beschreibung}: Synthese-Metriken könnten zu komplex oder ungenau sein
  \item \textbf{Wahrscheinlichkeit}: Mittel
  \item \textbf{Auswirkung}: Mittel (kann Genauigkeit beeinträchtigen)
\end{itemize}

\subsubsection{Gegenmaßnahmen}

\begin{itemize}
  \item \textbf{Validierung}: Kontinuierliche Validierung gegen analytische Modelle
  \item \textbf{Benchmarks}: Vergleich mit bekannten Benchmarks
  \item \textbf{Iterative Verbesserung}: Kontinuierliche Verbesserung der Metriken
  \item \textbf{Dokumentation}: Detaillierte Dokumentation der Annahmen
\end{itemize}

\section{Reproduzierbarkeit}

\subsection{Simulations-Reproduzierbarkeit}

\subsubsection{Risiko}

\begin{itemize}
  \item \textbf{Beschreibung}: Simulationen könnten nicht reproduzierbar sein
  \item \textbf{Wahrscheinlichkeit}: Niedrig
  \item \textbf{Auswirkung}: Mittel (kann Validierung erschweren)
\end{itemize}

\subsubsection{Gegenmaßnahmen}

\begin{itemize}
  \item \textbf{Seed-Management}: Fixierung von Zufalls-Seeds
  \item \textbf{Deterministische Generierung}: Deterministische Code-Generierung
  \item \textbf{Versionierung}: Versionierung aller Konfigurationen
  \item \textbf{CI}: Automatisierte Tests in CI-Pipelines
  \item \textbf{Dokumentation}: Dokumentation der Umgebung und Konfiguration
\end{itemize}

\subsection{Umgebungs-Abhängigkeit}

\subsubsection{Risiko}

\begin{itemize}
  \item \textbf{Beschreibung}: Ergebnisse könnten von der Umgebung abhängen
  \item \textbf{Wahrscheinlichkeit}: Niedrig
  \item \textbf{Auswirkung}: Niedrig
\end{itemize}

\subsubsection{Gegenmaßnahmen}

\begin{itemize}
  \item \textbf{Containerisierung}: Nutzung von Containern für Reproduzierbarkeit
  \item \textbf{Dependency-Pinning}: Fixierung aller Abhängigkeiten
  \item \textbf{Environment-Dokumentation}: Vollständige Dokumentation der Umgebung
\end{itemize}

\section{Zeitplan-Risiken}

\subsection{Verzögerungen}

\subsubsection{Risiko}

\begin{itemize}
  \item \textbf{Beschreibung}: Unvorhergesehene Probleme könnten zu Verzögerungen führen
  \item \textbf{Wahrscheinlichkeit}: Mittel
  \item \textbf{Auswirkung}: Mittel (kann Meilensteine gefährden)
\end{itemize}

\subsubsection{Gegenmaßnahmen}

\begin{itemize}
  \item \textbf{Pufferzeiten}: Eingebaute Pufferzeiten im Zeitplan
  \item \textbf{Priorisierung}: Fokus auf kritische Funktionen
  \item \textbf{Agile Methoden}: Flexible Anpassung des Plans
  \item \textbf{Frühe Risiko-Erkennung}: Regelmäßige Risiko-Reviews
\end{itemize}

\section{Erweiterte Risiko-Analyse}

\subsection{Quantitative Risiko-Bewertung}

Neben der qualitativen Risiko-Bewertung können Risiken auch quantitativ bewertet werden, um eine fundiertere Entscheidungsgrundlage zu schaffen.

\subsubsection{Risiko-Score}

Der Risiko-Score wird berechnet als:

\begin{equation}
R = P \times I
\end{equation}

wobei:
\begin{itemize}
  \item $P$: Wahrscheinlichkeit des Eintretens (0-1)
  \item $I$: Auswirkung (1-5 Skala)
  \item $R$: Risiko-Score (0-5)
\end{itemize}

Risiken mit $R > 3$ werden als hoch eingestuft und erfordern sofortige Gegenmaßnahmen.

\subsubsection{Beispiel: Risiko-Bewertung für Toolkopplung}

\begin{itemize}
  \item \textbf{Wahrscheinlichkeit}: $P = 0.6$ (Mittel - Exportformate können sich ändern)
  \item \textbf{Auswirkung}: $I = 4$ (Hoch - kann Projekt verzögern)
  \item \textbf{Risiko-Score}: $R = 0.6 \times 4 = 2.4$ (Mittel-Hoch)
  \item \textbf{Bewertung}: Erfordert proaktive Gegenmaßnahmen
\end{itemize}

\subsection{Risiko-Monitoring}

Risiken müssen kontinuierlich überwacht werden, um frühzeitig auf Änderungen reagieren zu können.

\subsubsection{Risiko-Reviews}

Regelmäßige Risiko-Reviews (z.\,B. wöchentlich) umfassen:

\begin{itemize}
  \item \textbf{Status-Check}: Überprüfung des aktuellen Status jedes Risikos
  \item \textbf{Wahrscheinlichkeits-Update}: Aktualisierung der Wahrscheinlichkeit basierend auf neuen Informationen
  \item \textbf{Auswirkungs-Update}: Aktualisierung der Auswirkung basierend auf Projektfortschritt
  \item \textbf{Gegenmaßnahmen-Review}: Überprüfung der Wirksamkeit der Gegenmaßnahmen
  \item \textbf{Neue Risiken}: Identifikation neuer Risiken
\end{itemize}

\subsubsection{Risiko-Dashboard}

Ein Risiko-Dashboard visualisiert den aktuellen Risiko-Status:

\begin{itemize}
  \item \textbf{Risiko-Matrix}: Visualisierung aller Risiken nach Wahrscheinlichkeit und Auswirkung
  \item \textbf{Trend-Analyse}: Entwicklung der Risiken über Zeit
  \item \textbf{Top-Risiken}: Liste der kritischsten Risiken
  \item \textbf{Gegenmaßnahmen-Status}: Status der geplanten Gegenmaßnahmen
\end{itemize}

\section{Erweiterte Risiko-Analyse-Methoden}

Dieser Abschnitt beschreibt erweiterte Methoden zur Risiko-Analyse.

\subsection{Monte-Carlo-Simulation}

Monte-Carlo-Simulation ermöglicht eine probabilistische Risiko-Analyse:

\begin{itemize}
  \item \textbf{Parameter-Variation}: Variation von Parametern mit Wahrscheinlichkeitsverteilungen
  \item \textbf{Simulation}: Mehrere tausend Simulationsläufe
  \item \textbf{Statistische Analyse}: Analyse der Ergebnisse (Mittelwert, Standardabweichung, Perzentile)
  \item \textbf{Risiko-Quantifizierung}: Quantifizierung von Risiken basierend auf Ergebnissen
\end{itemize}

\subsection{Fault-Tree-Analysis}

Fault-Tree-Analysis ermöglicht eine systematische Analyse von Fehlerursachen:

\begin{itemize}
  \item \textbf{Fault-Tree}: Hierarchische Darstellung von Fehlerursachen
  \item \textbf{Minimal-Cut-Sets}: Minimale Kombinationen von Fehlern, die zu einem Systemausfall führen
  \item \textbf{Wahrscheinlichkeits-Berechnung}: Berechnung der Wahrscheinlichkeit von Systemausfällen
  \item \textbf{Optimierung}: Identifikation kritischer Pfade für Optimierung
\end{itemize}

\section{Risiko-Matrix}

\subsection{Bewertung}

\begin{table}[h]
  \centering
  \caption{Risiko-Matrix}
  \begin{tabular}{llll}
    \toprule
    Risiko & Wahrscheinlichkeit & Auswirkung & Priorität \\
    \midrule
    Toolkopplung & Mittel & Hoch & Hoch \\
    Fehlende Daten & Hoch & Mittel & Hoch \\
    Architektur-Komplexität & Mittel & Hoch & Hoch \\
    Simulationsplattform & Mittel & Mittel & Mittel \\
    Datenqualität & Mittel & Mittel & Mittel \\
    Metrik-Komplexität & Mittel & Mittel & Mittel \\
    Reproduzierbarkeit & Niedrig & Mittel & Niedrig \\
    Verzögerungen & Mittel & Mittel & Mittel \\
    \bottomrule
  \end{tabular}
  \label{tab:risiko_matrix}
\end{table}

\section{Risiko-Monitoring}

\subsection{Regelmäßige Reviews}

\begin{itemize}
  \item \textbf{Wöchentliche Reviews}: Wöchentliche Überprüfung der Risiken
  \item \textbf{Meilenstein-Reviews}: Detaillierte Reviews an Meilensteinen
  \item \textbf{Risiko-Register}: Dokumentation aller Risiken und Maßnahmen
\end{itemize}

\subsection{Eskalations-Prozess}

\begin{itemize}
  \item \textbf{Level 1}: Projektintern lösen
  \item \textbf{Level 2}: Mit Stakeholdern besprechen
  \item \textbf{Level 3}: Externe Unterstützung einholen
\end{itemize}

\section{Zusammenfassung}

Dieses Kapitel hat die Risiken und Gegenmaßnahmen beschrieben:

\begin{itemize}
  \item \textbf{Technische Risiken}: Toolkopplung, Simulationsplattform
  \item \textbf{Datenrisiken}: Fehlende/unvollständige Daten, Datenqualität
  \item \textbf{Komplexitäts-Risiken}: Architektur- und Metrik-Komplexität
  \item \textbf{Reproduzierbarkeits-Risiken}: Simulations- und Umgebungs-Abhängigkeit
  \item \textbf{Zeitplan-Risiken}: Verzögerungen
\end{itemize}

Die geplanten Gegenmaßnahmen (IM-Schicht, Benchmarks, inkrementelle Schritte, CI) helfen, diese Risiken zu minimieren und das Projekt erfolgreich abzuschließen.
