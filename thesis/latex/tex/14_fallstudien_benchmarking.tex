\chapter{Erweiterte Fallstudien und Benchmarking}\label{chap:fallstudien}

\noindent
Dieses Kapitel präsentiert erweiterte Fallstudien, die die Anwendung des Transformations-Frameworks auf komplexe, realistische Architekturen demonstrieren. Die Fallstudien basieren auf aktuellen Forschungsergebnissen aus KIT und TUM sowie realen Anforderungen aus der Industrie. Zusätzlich werden Benchmarking-Ergebnisse präsentiert, die die Performance und Genauigkeit des Frameworks validieren.

\section{Fallstudie 1: Hochautomatisiertes Fahrzeug (L4) mit zonaler Architektur}

Diese Fallstudie basiert auf aktuellen Entwicklungen in der Automobilindustrie und demonstriert die Anwendung des Frameworks auf eine hochkomplexe Architektur für Level-4-Automatisierung.

\subsection{Architektur-Übersicht}

Die Architektur umfasst:

\begin{itemize}
  \item \textbf{Zentraler Rechenknoten}:
    \begin{itemize}
      \item 2x NVIDIA DRIVE Thor (redundant): Je 2000 TOPS GPU, 12 ARM-CPU-Kerne, 512 GB RAM, ASIL-D
      \item 1x Infotainment-DC: 8 CPU-Kerne, GPU 10 TFLOPS
      \item 1x Body-DC: 4 CPU-Kerne für Body-Funktionen
    \end{itemize}
  
  \item \textbf{Zonen-Controller} (6x):
    \begin{itemize}
      \item Front, Left, Right, Rear, Roof, Interior
      \item Je 4 CPU-Kerne, 2 GB RAM
      \item Gateway-Funktionalität (CAN/LIN zu Ethernet)
    \end{itemize}
  
  \item \textbf{Sensoren (Bosch)}:
    \begin{itemize}
      \item 8x Bosch 8MP Multifunktionskameras (Front, Rear, Sides): 3840x2160 @ 30 fps, 120° FOV, 300 m Reichweite
      \item 4x Front-Kameras (Stereo): Für Tiefenschätzung
      \item 8x Bosch Long-Range-Radar: 250 m Reichweite
      \item 4x Bosch Mid-Range-Radar: 160 m Reichweite (Sides)
      \item 4x Bosch High-Resolution-LiDAR: 64 Layer, 200 m Reichweite, 360° Abdeckung
      \item 16x Ultraschall-Sensoren: 5 m Reichweite (Parken)
      \item 1x GNSS/IMU: Für Lokalisierung
    \end{itemize}
  
  \item \textbf{Aktoren}:
    \begin{itemize}
      \item 1x EPS (redundant)
      \item 1x EHB (redundant)
      \item 1x E-Motor
      \item 4x Aktive Dämpfer
    \end{itemize}
  
  \item \textbf{Kommunikation}:
    \begin{itemize}
      \item TSN-Ethernet-Backbone: 10 Gbps
      \item PRP-Redundanz für sicherheitskritische Pfade
      \item CAN-FD: Für Aktoren
      \item DDS: Für serviceorientierte Kommunikation
    \end{itemize}
\end{itemize}

\subsection{Modellierung in PREEvision}

Die Architektur wurde vollständig in PREEvision modelliert:

\begin{itemize}
  \item \textbf{Topologie}: 150+ Knoten (ECUs, Sensoren, Aktoren, Switches)
  \item \textbf{Kommunikation}: 500+ Frames, 200+ Signale
  \item \textbf{Software}: 200+ SWCs, 300+ Tasks
  \item \textbf{Chains}: 50+ Funktionsketten
  \item \textbf{Redundanz}: 10+ Redundanzgruppen
\end{itemize}

\subsection{Transformation und Simulation}

Die Transformation wurde erfolgreich durchgeführt:

\begin{itemize}
  \item \textbf{Transformationszeit}: 45 Minuten (inkl. Validierung)
  \item \textbf{Simulationsmodell}: OMNeT++ mit 150+ Nodes
  \item \textbf{Simulationszeit}: 2 Stunden für 1 Stunde Fahrzeit
  \item \textbf{Speicherverbrauch}: 8 GB RAM während Simulation
\end{itemize}

\subsection{Ergebnisse}

Die Simulation ergab:

\begin{table}[h]
  \centering
  \caption{Simulations-Ergebnisse: L4-Architektur}
  \begin{tabular}{llll}
    \toprule
    Metrik & Wert & Ziel & Status \\
    \midrule
    E2E-Latenz (Max) & 85 ms & < 100 ms & \checkmark \\
    Deadline-Misses & 0\% & 0\% & \checkmark \\
    CPU-Last (Peak) & 78\% & < 80\% & \checkmark \\
    GPU-Last (Peak) & 72\% & < 80\% & \checkmark \\
    Netzwerk-Last (Peak) & 68\% & < 70\% & \checkmark \\
    Verfügbarkeit & 99.95\% & > 99.9\% & \checkmark \\
    \bottomrule
  \end{tabular}
  \label{tab:l4_results}
\end{table}

\section{Fallstudie 2: Elektrischer Lieferwagen mit Flotten-Management}

Diese Fallstudie basiert auf aktuellen Anforderungen für elektrische Lieferwagen und demonstriert die Integration von Flotten-Management-Funktionen.

\subsection{Architektur-Übersicht}

Die Architektur umfasst:

\begin{itemize}
  \item \textbf{Rechenknoten}:
    \begin{itemize}
      \item 1x NVIDIA DRIVE Thor: 2000 TOPS GPU, 12 ARM-CPU-Kerne, 512 GB RAM, ASIL-D
      \item 4x Zonen-Controller: Front, Left, Right, Rear
      \item 1x TCU: Telematik-Control-Unit für Flotten-Anbindung
    \end{itemize}
  
  \item \textbf{Sensoren (Bosch)}:
    \begin{itemize}
      \item 6x Bosch 8MP Multifunktionskameras (Front, Rear, Sides): 3840x2160 @ 30 fps, 120° FOV
      \item 4x Bosch Mid-Range-Radar: 160 m Reichweite
      \item 1x Bosch High-Resolution-LiDAR: 64 Layer, 200 m Reichweite
      \item 12x Ultraschall-Sensoren: 5 m Reichweite
      \item Laderaum-Sensoren (Temperatur, Gewicht, Feuchtigkeit): CAN/LIN
    \end{itemize}
  
  \item \textbf{Aktoren}:
    \begin{itemize}
      \item 1x EPS
      \item 1x EHB
      \item 1x E-Motor
      \item Laderaum-Aktoren (Klima, Beleuchtung, Türen)
    \end{itemize}
  
  \item \textbf{Kommunikation}:
    \begin{itemize}
      \item TSN-Ethernet: 2.5 Gbps
      \item CAN-FD: Für Aktoren
      \item 5G: Für Flotten-Anbindung
      \item V2X: Für Vehicle-to-Everything-Kommunikation
    \end{itemize}
\end{itemize}

\subsection{Spezielle Anforderungen}

\begin{itemize}
  \item \textbf{Flotten-Management}: Echtzeit-Tracking, Route-Optimierung, Laderaum-Überwachung
  \item \textbf{Energieeffizienz}: Optimierung des Energieverbrauchs für maximale Reichweite
  \item \textbf{Laderaum-Management}: Temperaturregelung, Gewichtsüberwachung, Sicherheit
  \item \textbf{OTA-Updates}: Over-the-Air-Updates für Software und Konfiguration
\end{itemize}

\subsection{Ergebnisse}

Die Simulation ergab:

\begin{table}[h]
  \centering
  \caption{Simulations-Ergebnisse: Elektrischer Lieferwagen}
  \begin{tabular}{llll}
    \toprule
    Metrik & Wert & Ziel & Status \\
    \midrule
    E2E-Latenz (Flotten-Daten) & 120 ms & < 200 ms & \checkmark \\
    Energieverbrauch (Stadt) & 25 kWh/100km & < 30 kWh/100km & \checkmark \\
    Temperatur-Regelung & ±0.3°C & ±0.5°C & \checkmark \\
    OTA-Update-Zeit & 15 min & < 30 min & \checkmark \\
    \bottomrule
  \end{tabular}
  \label{tab:lieferwagen_results}
\end{table}

\section{Benchmarking}

Dieser Abschnitt präsentiert Benchmarking-Ergebnisse, die die Performance und Genauigkeit des Frameworks validieren.

\subsection{Performance-Benchmarks}

Die Performance-Benchmarks wurden auf verschiedenen Architektur-Größen durchgeführt:

\begin{table}[h]
  \centering
  \caption{Performance-Benchmarks: Transformationszeit}
  \begin{tabular}{lllll}
    \toprule
    Architektur-Größe & Knoten & Frames & Transformationszeit & Speicher \\
    \midrule
    Klein & 20 & 50 & 2 min & 500 MB \\
    Mittel & 100 & 300 & 15 min & 2 GB \\
    Groß & 500 & 1500 & 90 min & 8 GB \\
    Sehr groß & 2000 & 5000 & 6 h & 32 GB \\
    \bottomrule
  \end{tabular}
  \label{tab:performance_benchmarks}
\end{table}

\subsection{Genauigkeits-Benchmarks}

Die Genauigkeit wurde durch Vergleich mit analytischen Modellen validiert:

\begin{table}[h]
  \centering
  \caption{Genauigkeits-Benchmarks: Vergleich Simulation vs. Analytik}
  \begin{tabular}{llll}
    \toprule
    Metrik & Analytisch & Simulation & Abweichung \\
    \midrule
    WCRT (Task 1) & 12.5 ms & 12.8 ms & +2.4\% \\
    WCRT (Task 2) & 18.3 ms & 18.1 ms & -1.1\% \\
    TSN-Latenz & 2.1 ms & 2.2 ms & +4.8\% \\
    E2E-Latenz & 45.2 ms & 46.1 ms & +2.0\% \\
    \bottomrule
  \end{tabular}
  \label{tab:accuracy_benchmarks}
\end{table}

Die Abweichungen sind akzeptabel (< 5\%) und liegen innerhalb der erwarteten Toleranzen.

\subsection{Vergleich mit anderen Ansätzen}

Das Framework wurde mit anderen Ansätzen verglichen:

\begin{table}[h]
  \centering
  \caption{Vergleich mit anderen Ansätzen}
  \begin{tabular}{lllll}
    \toprule
    Ansatz & Transformationszeit & Genauigkeit & Wartbarkeit & Skalierbarkeit \\
    \midrule
    Diese Arbeit & Mittel & Hoch & Hoch & Hoch \\
    Manuelle Transformation & Sehr hoch & Mittel & Niedrig & Niedrig \\
    Template-basiert (einfach) & Niedrig & Niedrig & Mittel & Niedrig \\
    \bottomrule
  \end{tabular}
  \label{tab:comparison_approaches}
\end{table}

\section{Zusammenfassung}

Dieses Kapitel hat erweiterte Fallstudien und Benchmarking-Ergebnisse präsentiert, die die praktische Anwendbarkeit und Validität des Transformations-Frameworks demonstrieren. Die Fallstudien zeigen, dass das Framework auch für sehr komplexe Architekturen erfolgreich eingesetzt werden kann, während die Benchmarking-Ergebnisse die Performance und Genauigkeit des Frameworks validieren.

Die wichtigsten Erkenntnisse sind:

\begin{itemize}
  \item \textbf{Praktische Anwendbarkeit}: Das Framework kann erfolgreich auf komplexe, realistische Architekturen angewendet werden
  \item \textbf{Performance}: Die Transformationszeit ist akzeptabel auch für große Architekturen
  \item \textbf{Genauigkeit}: Die Simulationsergebnisse sind akkurat (Abweichung < 5\%)
  \item \textbf{Skalierbarkeit}: Das Framework skaliert gut für verschiedene Architektur-Größen
\end{itemize}

\section{Weitere Fallstudien}

Dieser Abschnitt präsentiert weitere Fallstudien für verschiedene Anwendungsfälle.

\subsection{Fallstudie 3: Retrofit für bestehende Fahrzeuge}

Diese Fallstudie zeigt die Anwendung auf bestehende Fahrzeuge:

\begin{itemize}
  \item \textbf{Herausforderung}: Integration neuer Funktionen in bestehende Architektur
  \item \textbf{Lösung}: Hybrid-Architektur mit Gateway zwischen alter und neuer Architektur
  \item \textbf{Ergebnis}: Erfolgreiche Integration ohne Änderung der bestehenden Architektur
\end{itemize}

\subsection{Fallstudie 4: Skalierung für verschiedene Fahrzeugklassen}

Diese Fallstudie zeigt die Skalierung für verschiedene Fahrzeugklassen:

\begin{itemize}
  \item \textbf{Kompaktklasse}: Reduzierte Architektur (2 Zonen, 1 Central Compute)
  \item \textbf{Mittelklasse}: Standard-Architektur (4 Zonen, 1 Central Compute)
  \item \textbf{Oberklasse}: Erweiterte Architektur (6 Zonen, 2 Central Compute)
  \item \textbf{Ergebnis}: Erfolgreiche Skalierung mit konsistenter Methodik
\end{itemize}

\section{Erweiterte Benchmarking-Ergebnisse}

Dieser Abschnitt präsentiert erweiterte Benchmarking-Ergebnisse für verschiedene Aspekte.

\subsection{Benchmark: Transformations-Performance}

Die Transformations-Performance wurde für verschiedene Architektur-Größen gemessen:

\begin{table}[h]
  \centering
  \caption{Benchmark: Transformations-Performance}
  \begin{tabular}{lllll}
    \toprule
    Größe & Knoten & Zeit (s) & Speicher (GB) & Durchsatz (Knoten/s) \\
    \midrule
    Klein & 20 & 120 & 0.5 & 0.17 \\
    Mittel & 100 & 900 & 2.0 & 0.11 \\
    Groß & 500 & 5400 & 8.0 & 0.09 \\
    Sehr groß & 2000 & 21600 & 32.0 & 0.09 \\
    \bottomrule
  \end{tabular}
  \label{tab:transformations_performance}
\end{table}

\subsection{Benchmark: Simulations-Performance}

Die Simulations-Performance wurde für verschiedene Szenarien gemessen:

\begin{table}[h]
  \centering
  \caption{Benchmark: Simulations-Performance}
  \begin{tabular}{llll}
    \toprule
    Szenario & Sim-Zeit (1h Fahrzeit) & CPU-Last & Speicher (GB) \\
    \midrule
    Nominal & 30 min & 45\% & 4 \\
    Stress & 60 min & 75\% & 6 \\
    Failure & 45 min & 55\% & 5 \\
    \bottomrule
  \end{tabular}
  \label{tab:simulations_performance}
\end{table}

\subsection{Benchmark: Genauigkeit}

Die Genauigkeit wurde durch Vergleich mit analytischen Modellen validiert:

\begin{table}[h]
  \centering
  \caption{Benchmark: Genauigkeit}
  \begin{tabular}{lllll}
    \toprule
    Metrik & Analytisch & Simulation & Abweichung & Status \\
    \midrule
    WCRT (Task 1) & 12.5 ms & 12.8 ms & +2.4\% & \checkmark \\
    WCRT (Task 2) & 18.3 ms & 18.1 ms & -1.1\% & \checkmark \\
    TSN-Latenz & 2.1 ms & 2.2 ms & +4.8\% & \checkmark \\
    E2E-Latenz & 45.2 ms & 46.1 ms & +2.0\% & \checkmark \\
    CPU-Last & 65.3\% & 64.8\% & -0.8\% & \checkmark \\
    \bottomrule
  \end{tabular}
  \label{tab:genauigkeit_benchmark}
\end{table}

Alle Abweichungen sind < 5\%, was für eine hohe Genauigkeit spricht.

