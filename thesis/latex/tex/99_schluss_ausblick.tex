\chapter{Schlussfolgerungen und Ausblick}

\section{Zusammenfassung der Beiträge}

Die Arbeit etabliert eine reproduzierbare, validierte Kette von der Architektur bis zur Simulation und zeigt, wie frühe, datengetriebene Architekturentscheidungen möglich werden. Die entwickelten Methoden und Werkzeuge ermöglichen es, komplexe E/E-Architekturen für moderne Fahrzeuge \cite{reif_ee_architektur, braess_seiffen_handbuch} frühzeitig zu evaluieren und zu optimieren, bevor kostspielige Hardware-Prototypen erstellt werden.

Die zentralen Beiträge dieser Arbeit umfassen:
\begin{itemize}
  \item Eine domänenspezifische Komponententaxonomie und ein erweiterbares Metamodell für moderne E/E-Architekturen, das die Anforderungen autonomer Fahrzeuge \cite{automated_driving_systems} und moderner Kommunikationstechnologien \cite{tsn_automotive, vehicle_networking} berücksichtigt.
  
  \item Eine methodische Erweiterung der Synthese-Metrik zur quantitativen Ableitung von Ressourcen-, Timing- und Verfügbarkeitsparametern, die auf etablierten Methoden der Echtzeit-Systeme \cite{real_time_systems} und funktionalen Sicherheit \cite{iso26262_practice} aufbaut.
  
  \item Ein regelbasiertes Transformationsframework, das moderne Simulationsplattformen \cite{network_simulation} und modellbasierte Entwicklungsansätze \cite{model_based_development, autosar_practice} unterstützt.
  
  \item Eine systematische Evaluationsmethodik, die moderne Validierungsansätze \cite{validation_verification, simulation_automotive} für Fahrzeugsysteme nutzt.
\end{itemize}

\section{Ausblick und zukünftige Arbeiten}

Künftige Arbeiten adressieren verschiedene Erweiterungen und Verbesserungen:

\subsection{Erweiterte Modellierung}

\begin{itemize}
  \item \textbf{Physikalische Sensor-/Aktormodelle}: Integration detaillierter physikalischer Modelle für Sensoren \cite{automotive_sensors} und Aktoren, um realistischere Simulationen zu ermöglichen.
  
  \item \textbf{Adaptive Scheduling-Strategien}: Entwicklung adaptiver Scheduling-Algorithmen, die sich dynamisch an Laständerungen anpassen \cite{real_time_systems}.
  
  \item \textbf{KI/ML-Integration}: Integration von KI-Modellen \cite{automotive_ai} in die Simulation, um realistischere Perzeptions- und Entscheidungsprozesse zu modellieren.
\end{itemize}

\subsection{Erweiterte Validierung}

\begin{itemize}
  \item \textbf{Closed-Loop-Verifikation}: Integration von HIL/SiL-Co-Simulation \cite{simulation_automotive} für Closed-Loop-Validierung.
  
  \item \textbf{Formale Verifikation}: Kombination von Simulation mit formalen Verifikationsmethoden \cite{validation_verification} für sicherheitskritische Systeme.
  
  \item \textbf{Continuous Validation}: Integration der Validierung in CI/CD-Pipelines für kontinuierliche Qualitätssicherung.
\end{itemize}

\subsection{Skalierung und Performance}

\begin{itemize}
  \item \textbf{Cloud-basierte Simulationen}: Nutzung von Cloud-Computing-Ressourcen für große Simulationsläufe.
  
  \item \textbf{Distributed Simulation}: Entwicklung verteilter Simulationsansätze für sehr große Architekturen.
  
  \item \textbf{Real-time Simulation}: Optimierung für Echtzeit-Simulationen für Hardware-in-the-Loop-Tests.
\end{itemize}

\subsection{Integration moderner Technologien}

\begin{itemize}
  \item \textbf{Edge-Cloud-Hybrid}: Unterstützung für Edge-Cloud-Hybrid-Architekturen \cite{cyber_physical_systems}.
  
  \item \textbf{5G/V2X}: Integration von 5G und V2X-Kommunikation \cite{vehicle_networking} in die Simulation.
  
  \item \textbf{Digital Twins}: Entwicklung von Digital-Twin-Ansätzen für kontinuierliche Validierung während des Fahrzeugbetriebs.
\end{itemize}

Die kontinuierliche Weiterentwicklung dieser Methoden wird entscheidend sein, um mit der wachsenden Komplexität moderner Fahrzeuge \cite{automotive_software, automotive_electronics} Schritt zu halten und die Entwicklung sicherer, effizienter und zuverlässiger E/E-Architekturen zu unterstützen.

\section{Lessons Learned und Best Practices}

Basierend auf der Entwicklung und Evaluierung des Transformations-Frameworks können mehrere wichtige Erkenntnisse und Best Practices abgeleitet werden, die für zukünftige Projekte von Bedeutung sind.

\subsection{Architektur-Design}

\begin{itemize}
  \item \textbf{Intermediate Model als Abstraktionsebene}: Die Verwendung eines Intermediate Models als Abstraktionsebene zwischen Quell- und Zielformaten hat sich als sehr wertvoll erwiesen. Es ermöglicht es, verschiedene Quellformate (nicht nur PREEvision) und Zielplattformen zu unterstützen, ohne die Transformationslogik zu duplizieren.
  
  \item \textbf{Schema-basierte Validierung}: Die Verwendung formaler Schemas (JSON Schema, XML Schema) für die Validierung von Modellen hat dazu beigetragen, Fehler frühzeitig zu erkennen und die Konsistenz der Modelle zu gewährleisten.
  
  \item \textbf{Erweiterbarkeit durch Stereotypen}: Der Stereotyp-Mechanismus ermöglicht es, domänenspezifische Erweiterungen vorzunehmen, ohne das Basis-Metamodell zu ändern. Dies erleichtert die Wartung und ermöglicht es, verschiedene Domänen zu unterstützen.
\end{itemize}

\subsection{Transformations-Strategie}

\begin{itemize}
  \item \textbf{Regelbasierte Transformation}: Die Verwendung regelbasierter Transformationen hat sich als flexibel und wartbar erwiesen. Regeln können unabhängig voneinander entwickelt, getestet und gewartet werden.
  
  \item \textbf{Template-basierte Code-Generierung}: Die Verwendung von Templates (z.\,B. Jinja2) für die Code-Generierung ermöglicht es, die Generierungslogik von der Transformationslogik zu trennen und erleichtert die Wartung.
  
  \item \textbf{Inkrementelles Vorgehen}: Der Minimalstart-Ansatz hat sich als sehr wertvoll erwiesen. Durch die frühzeitige Implementierung einer einfachen, aber vollständigen Architektur konnten Probleme frühzeitig erkannt und behoben werden.
\end{itemize}

\subsection{Validierung und Qualitätssicherung}

\begin{itemize}
  \item \textbf{Mehrschichtige Validierung}: Die Validierung auf mehreren Ebenen (Syntax, Semantik, Constraints) hat dazu beigetragen, Fehler in verschiedenen Phasen zu erkennen.
  
  \item \textbf{Analytische Modelle als Referenz}: Der Vergleich mit analytischen Modellen hat dazu beigetragen, die Korrektheit der Simulationsergebnisse zu validieren.
  
  \item \textbf{Automatisierte Tests}: Die Verwendung automatisierter Tests (Unit-Tests, Integration-Tests) hat die Qualität des Codes erheblich verbessert und Regressionen verhindert.
\end{itemize}

\subsection{Projekt-Management}

\begin{itemize}
  \item \textbf{Frühe Stakeholder-Einbindung}: Die frühzeitige Einbindung von Stakeholdern (Domänenexperten, Tool-Experten) hat dazu beigetragen, Anforderungen zu klären und Fehlentwicklungen zu vermeiden.
  
  \item \textbf{Iterative Entwicklung}: Die iterative Entwicklung mit regelmäßigen Reviews und Anpassungen hat die Qualität der Ergebnisse verbessert und Risiken reduziert.
  
  \item \textbf{Dokumentation als First-Class Citizen}: Die frühzeitige und kontinuierliche Dokumentation hat dazu beigetragen, Wissen zu bewahren und die Wartbarkeit zu verbessern.
\end{itemize}

\section{Beitrag zur Wissenschaft und Praxis}

Diese Arbeit leistet mehrere wichtige Beiträge zur Wissenschaft und Praxis der E/E-Architektur-Entwicklung:

\subsection{Wissenschaftliche Beiträge}

\begin{itemize}
  \item \textbf{Metamodell für E/E-Architekturen}: Entwicklung eines erweiterbaren Metamodells, das die Komplexität moderner E/E-Architekturen abbildet und für verschiedene Anwendungsfälle verwendet werden kann.
  
  \item \textbf{Synthese-Metrik}: Erweiterung der Synthese-Metrik zur automatisierten Ableitung von Simulationsparametern aus Architekturmerkmalen, was die Effizienz der Architektur-Evaluierung verbessert.
  
  \item \textbf{Transformations-Framework}: Entwicklung eines regelbasierten Transformations-Frameworks, das die Transformation von Architekturmodellen in Simulationsmodelle automatisiert und reproduzierbar macht.
  
  \item \textbf{Validierungsmethodik}: Entwicklung einer systematischen Methodik zur Validierung von Simulationsergebnissen durch Vergleich mit analytischen Modellen und Sensitivitätsanalysen.
\end{itemize}

\subsection{Praktische Beiträge}

\begin{itemize}
  \item \textbf{Open-Source-Framework}: Bereitstellung eines Open-Source-Frameworks, das von anderen Entwicklern verwendet und erweitert werden kann.
  
  \item \textbf{Best Practices}: Dokumentation von Best Practices und Lessons Learned, die für zukünftige Projekte von Nutzen sind.
  
  \item \textbf{Beispiel-Architekturen}: Bereitstellung von Beispiel-Architekturen und Simulations-Szenarien, die als Referenz für andere Entwickler dienen können.
  
  \item \textbf{Integration in Entwicklungsworkflow}: Demonstration der Integration von Simulation in den Architektur-Entwicklungsworkflow, was die Effizienz der Entwicklung verbessert.
\end{itemize}

\section{Ausblick auf zukünftige Entwicklungen}

Die Entwicklung von E/E-Architekturen wird in den kommenden Jahren weiterhin von mehreren Trends geprägt sein, die neue Anforderungen an Simulations- und Validierungsmethoden stellen:

\subsection{Technologische Trends}

\begin{itemize}
  \item \textbf{Zunehmende Komplexität}: Die Komplexität von E/E-Architekturen wird weiter zunehmen, was leistungsfähigere Simulations- und Validierungsmethoden erfordert.
  
  \item \textbf{KI/ML-Integration}: Die Integration von KI/ML-Modellen in E/E-Architekturen erfordert neue Simulationsansätze, die diese Modelle berücksichtigen.
  
  \item \textbf{Edge-Cloud-Hybrid}: Die Entwicklung von Edge-Cloud-Hybrid-Architekturen erfordert neue Simulationsansätze, die sowohl Edge- als auch Cloud-Komponenten modellieren.
  
  \item \textbf{5G/V2X}: Die Integration von 5G und V2X-Kommunikation erfordert neue Simulationsansätze für drahtlose Kommunikation.
\end{itemize}

\subsection{Methodische Trends}

\begin{itemize}
  \item \textbf{Digital Twins}: Die Entwicklung von Digital-Twin-Ansätzen ermöglicht kontinuierliche Validierung während des Fahrzeugbetriebs.
  
  \item \textbf{Continuous Validation}: Die Integration von Validierung in CI/CD-Pipelines ermöglicht kontinuierliche Qualitätssicherung.
  
  \item \textbf{Formale Verifikation}: Die Kombination von Simulation mit formaler Verifikation ermöglicht höhere Sicherheitsgarantien.
  
  \item \textbf{AI-basierte Optimierung}: Die Verwendung von KI für die automatische Optimierung von Architekturen wird zunehmend relevant.
\end{itemize}

Diese Trends werden die Entwicklung von E/E-Architekturen in den kommenden Jahren prägen und neue Anforderungen an Simulations- und Validierungsmethoden stellen. Die in dieser Arbeit entwickelten Methoden und Werkzeuge bilden eine solide Grundlage für die Bewältigung dieser Herausforderungen.

\section{Zusammenfassung der Beiträge}

Diese Arbeit leistet mehrere wichtige Beiträge zur Forschung und Praxis der E/E-Architektur-Entwicklung:

\subsection{Methodische Beiträge}

\begin{itemize}
  \item \textbf{Erweiterbares Metamodell}: Entwicklung eines erweiterbaren Metamodells für moderne E/E-Architekturen, das zonale Architekturen, TSN-Netzwerke, KI-Integration und Cybersecurity-Aspekte abdeckt.
  
  \item \textbf{Synthese-Metrik}: Erweiterte Synthese-Metrik zur automatisierten Ableitung von Simulationsparametern aus Architekturmerkmalen, einschließlich Multi-Core-Scheduling, TSN-Latenz-Berechnung und Energie-Modellierung.
  
  \item \textbf{Transformations-Framework}: Regelbasiertes Transformations-Framework für die automatische Transformation von Architekturmodellen in Simulationsmodelle, unterstützt durch ein Intermediate Model.
  
  \item \textbf{Validierungsmethodik}: Systematische Validierungsmethodik mit analytischen Modellen, Sensitivitätsanalysen und Benchmarking.
\end{itemize}

\subsection{Praktische Beiträge}

\begin{itemize}
  \item \textbf{Open-Source-Framework}: Bereitstellung eines Open-Source-Frameworks für die Transformation und Simulation von E/E-Architekturen.
  
  \item \textbf{Best Practices}: Dokumentation von Best Practices und Lessons Learned für die praktische Anwendung.
  
  \item \textbf{Fallstudien}: Umfassende Fallstudien, die die praktische Anwendbarkeit demonstrieren.
  
  \item \textbf{Benchmarking}: Benchmarking-Ergebnisse, die die Performance und Genauigkeit validieren.
\end{itemize}

\section{Fazit}

Diese Arbeit hat eine umfassende Methodik zur Transformation von E/E-Architekturmodellen in Simulationsmodelle entwickelt und validiert. Die Methodik ermöglicht es, Architekturentscheidungen frühzeitig zu validieren und Optimierungspotenziale zu identifizieren, bevor kostspielige Hardware-Prototypen erstellt werden.

Die entwickelten Methoden und Werkzeuge bilden eine solide Grundlage für die Bewältigung der Herausforderungen moderner E/E-Architekturen und können einen wichtigen Beitrag zur Effizienz und Qualität der Fahrzeugentwicklung leisten.

\section{Erweiterte Lessons Learned}

Dieser Abschnitt beschreibt erweiterte Lessons Learned aus der Entwicklung und Anwendung des Frameworks.

\subsection{Technische Lessons Learned}

\begin{itemize}
  \item \textbf{Metamodell-Design}: Ein gut durchdachtes Metamodell ist entscheidend für die Erweiterbarkeit
  \item \textbf{Performance}: Streaming-Parsing ist essentiell für große Modelle
  \item \textbf{Validierung}: Frühe und kontinuierliche Validierung spart Zeit
  \item \textbf{Modularität}: Modulares Design ermöglicht einfache Erweiterung
\end{itemize}

\subsection{Methodische Lessons Learned}

\begin{itemize}
  \item \textbf{Inkrementeller Ansatz}: Inkrementeller Ansatz reduziert Risiken
  \item \textbf{Feedback-Loops}: Kurze Feedback-Loops verbessern die Qualität
  \item \textbf{Dokumentation}: Umfassende Dokumentation ist essentiell
  \item \textbf{Beispiele}: Praktische Beispiele erleichtern das Verständnis
\end{itemize}

\subsection{Organisatorische Lessons Learned}

\begin{itemize}
  \item \textbf{Stakeholder-Einbindung}: Frühe Einbindung von Stakeholdern ist wichtig
  \item \textbf{Kommunikation}: Regelmäßige Kommunikation verhindert Missverständnisse
  \item \textbf{Risikomanagement}: Proaktives Risikomanagement ist essentiell
  \item \textbf{Qualitätssicherung}: Kontinuierliche Qualitätssicherung spart Zeit
\end{itemize}


