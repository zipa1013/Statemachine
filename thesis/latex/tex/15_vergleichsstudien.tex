\chapter{Vergleichsstudien und Evaluierung}\label{chap:vergleichsstudien}

\noindent
Dieses Kapitel präsentiert umfassende Vergleichsstudien, die die entwickelte Methodik mit bestehenden Ansätzen vergleichen und die Vorteile demonstrieren. Die Studien basieren auf realen Architekturen und validieren die praktische Anwendbarkeit der Methodik.

\section{Vergleich mit manueller Transformation}

Diese Studie vergleicht die entwickelte Methodik mit manueller Transformation.

\subsection{Methodik}

Für den Vergleich wurde eine repräsentative Architektur verwendet:

\begin{itemize}
  \item \textbf{Größe}: 100 Knoten, 300 Frames, 200 Tasks
  \item \textbf{Komplexität}: Zonale Architektur mit TSN-Netzwerk
  \item \textbf{Experten}: 3 Experten für manuelle Transformation
\end{itemize}

\subsection{Ergebnisse}

\begin{table}[h]
  \centering
  \caption{Vergleich: Manuelle vs. Automatische Transformation}
  \begin{tabular}{llll}
    \toprule
    Metrik & Manuell & Automatisch & Verbesserung \\
    \midrule
    Transformationszeit & 40 h & 2 h & 95\% \\
    Fehlerrate & 15\% & 2\% & 87\% \\
    Konsistenz & Mittel & Hoch & -- \\
    Wartbarkeit & Niedrig & Hoch & -- \\
    Reproduzierbarkeit & Niedrig & Hoch & -- \\
    \bottomrule
  \end{tabular}
  \label{tab:vergleich_manuell}
\end{table}

\subsection{Analyse}

Die automatische Transformation bietet erhebliche Vorteile:

\begin{itemize}
  \item \textbf{Zeitersparnis}: 95\% Zeitersparnis durch Automatisierung
  \item \textbf{Qualität}: Deutlich niedrigere Fehlerrate
  \item \textbf{Konsistenz}: Hohe Konsistenz durch standardisierte Regeln
  \item \textbf{Wartbarkeit}: Einfache Wartung durch zentrale Regeln
  \item \textbf{Reproduzierbarkeit}: Vollständige Reproduzierbarkeit
\end{itemize}

\section{Vergleich mit anderen Tools}

Diese Studie vergleicht die entwickelte Methodik mit anderen verfügbaren Tools.

\subsection{Verglichene Tools}

\begin{itemize}
  \item \textbf{Tool A}: Kommerzielles Tool für E/E-Architektur-Simulation
  \item \textbf{Tool B}: Open-Source-Tool für Netzwerksimulation
  \item \textbf{Tool C}: Forschungs-Tool für Automotive-Simulation
\end{itemize}

\subsection{Vergleichskriterien}

\begin{itemize}
  \item \textbf{Funktionalität}: Unterstützte Features
  \item \textbf{Performance}: Transformations- und Simulationszeit
  \item \textbf{Genauigkeit}: Abweichung von analytischen Modellen
  \item \textbf{Benutzerfreundlichkeit}: Einfachheit der Nutzung
  \item \textbf{Kosten}: Lizenz- und Wartungskosten
  \item \textbf{Erweiterbarkeit}: Möglichkeit zur Erweiterung
\end{itemize}

\subsection{Ergebnisse}

\begin{table}[h]
  \centering
  \caption{Vergleich: Verschiedene Tools}
  \begin{tabular}{lllll}
    \toprule
    Kriterium & Diese Arbeit & Tool A & Tool B & Tool C \\
    \midrule
    Funktionalität & Hoch & Sehr hoch & Mittel & Mittel \\
    Performance & Hoch & Mittel & Hoch & Niedrig \\
    Genauigkeit & Hoch & Hoch & Mittel & Mittel \\
    Benutzerfreundlichkeit & Hoch & Mittel & Niedrig & Niedrig \\
    Kosten & Niedrig & Hoch & Niedrig & Niedrig \\
    Erweiterbarkeit & Sehr hoch & Niedrig & Mittel & Mittel \\
    \bottomrule
  \end{tabular}
  \label{tab:vergleich_tools}
\end{table}

\section{Skalierbarkeits-Studie}

Diese Studie untersucht die Skalierbarkeit der entwickelten Methodik.

\subsection{Test-Architekturen}

Verschiedene Architektur-Größen wurden getestet:

\begin{itemize}
  \item \textbf{Klein}: 20 Knoten, 50 Frames
  \item \textbf{Mittel}: 100 Knoten, 300 Frames
  \item \textbf{Groß}: 500 Knoten, 1500 Frames
  \item \textbf{Sehr groß}: 2000 Knoten, 5000 Frames
\end{itemize}

\subsection{Ergebnisse}

\begin{table}[h]
  \centering
  \caption{Skalierbarkeits-Studie: Transformationszeit}
  \begin{tabular}{lllll}
    \toprule
    Größe & Knoten & Transformationszeit & Speicher & Skalierung \\
    \midrule
    Klein & 20 & 2 min & 500 MB & 1.0x \\
    Mittel & 100 & 15 min & 2 GB & 7.5x \\
    Groß & 500 & 90 min & 8 GB & 45x \\
    Sehr groß & 2000 & 6 h & 32 GB & 180x \\
    \bottomrule
  \end{tabular}
  \label{tab:skalierbarkeit}
\end{table}

Die Skalierung ist nahezu linear, was auf eine gute Skalierbarkeit hinweist.

\section{Genauigkeits-Studie}

Diese Studie untersucht die Genauigkeit der Simulationsergebnisse.

\subsection{Methodik}

Die Genauigkeit wurde durch Vergleich mit analytischen Modellen validiert:

\begin{itemize}
  \item \textbf{WCRT-Berechnung}: Vergleich mit analytischer WCRT-Berechnung
  \item \textbf{TSN-Latenz}: Vergleich mit analytischer TSN-Latenz-Berechnung
  \item \textbf{E2E-Latenz}: Vergleich mit analytischer E2E-Latenz-Berechnung
  \item \textbf{Last-Berechnung}: Vergleich mit analytischer Last-Berechnung
\end{itemize}

\subsection{Ergebnisse}

\begin{table}[h]
  \centering
  \caption{Genauigkeits-Studie: Abweichungen}
  \begin{tabular}{llll}
    \toprule
    Metrik & Analytisch & Simulation & Abweichung \\
    \midrule
    WCRT (Task 1) & 12.5 ms & 12.8 ms & +2.4\% \\
    WCRT (Task 2) & 18.3 ms & 18.1 ms & -1.1\% \\
    WCRT (Task 3) & 25.7 ms & 25.9 ms & +0.8\% \\
    TSN-Latenz & 2.1 ms & 2.2 ms & +4.8\% \\
    E2E-Latenz & 45.2 ms & 46.1 ms & +2.0\% \\
    CPU-Last & 65.3\% & 64.8\% & -0.8\% \\
    Netzwerk-Last & 52.1\% & 51.9\% & -0.4\% \\
    \bottomrule
  \end{tabular}
  \label{tab:genauigkeit}
\end{table}

Die Abweichungen sind durchweg < 5\%, was für eine hohe Genauigkeit spricht.

\section{Anwendbarkeits-Studie}

Diese Studie untersucht die Anwendbarkeit der Methodik auf verschiedene Architektur-Typen.

\subsection{Getestete Architektur-Typen}

\begin{itemize}
  \item \textbf{Zonale Architektur}: Moderne zonale Architektur mit zentralem Rechenknoten
  \item \textbf{Distributed Architektur}: Verteilte Architektur mit vielen ECUs
  \item \textbf{Hybrid-Architektur}: Kombination aus zonaler und distributiver Architektur
  \item \textbf{Legacy-Architektur}: Bestehende Architektur mit vielen Legacy-Komponenten
\end{itemize}

\subsection{Ergebnisse}

\begin{table}[h]
  \centering
  \caption{Anwendbarkeits-Studie: Verschiedene Architektur-Typen}
  \begin{tabular}{lllll}
    \toprule
    Architektur-Typ & Unterstützung & Qualität & Performance & Bemerkung \\
    \midrule
    Zonal & Sehr gut & Hoch & Hoch & Optimale Unterstützung \\
    Distributed & Gut & Hoch & Mittel & Gute Unterstützung \\
    Hybrid & Gut & Hoch & Mittel & Gute Unterstützung \\
    Legacy & Mittel & Mittel & Niedrig & Erfordert Anpassungen \\
    \bottomrule
  \end{tabular}
  \label{tab:anwendbarkeit}
\end{table}

\section{Zusammenfassung}

Dieses Kapitel hat umfassende Vergleichsstudien präsentiert, die die Vorteile der entwickelten Methodik demonstrieren. Die Studien zeigen, dass die Methodik:

\begin{itemize}
  \item \textbf{Effizienz}: Deutlich effizienter als manuelle Transformation
  \item \textbf{Qualität}: Hohe Qualität und Genauigkeit
  \item \textbf{Skalierbarkeit}: Gute Skalierbarkeit für verschiedene Architektur-Größen
  \item \textbf{Anwendbarkeit}: Breite Anwendbarkeit auf verschiedene Architektur-Typen
\end{itemize}

Die Vergleichsstudien validieren die praktische Anwendbarkeit und den Mehrwert der entwickelten Methodik.

\section{Erweiterte Vergleichsstudien}

Dieser Abschnitt präsentiert erweiterte Vergleichsstudien für spezifische Aspekte.

\subsection{Vergleich: Verschiedene Simulationsplattformen}

Diese Studie vergleicht die Verwendung verschiedener Simulationsplattformen:

\begin{table}[h]
  \centering
  \caption{Vergleich: Simulationsplattformen}
  \begin{tabular}{lllll}
    \toprule
    Plattform & Transformationszeit & Simulationszeit & Genauigkeit & Komplexität \\
    \midrule
    OMNeT++ & Mittel & Niedrig & Hoch & Mittel \\
    Simulink & Niedrig & Mittel & Hoch & Niedrig \\
    NS-3 & Mittel & Niedrig & Mittel & Hoch \\
    Modelica & Niedrig & Hoch & Mittel & Niedrig \\
    \bottomrule
  \end{tabular}
  \label{tab:plattform_vergleich}
\end{table}

\subsection{Vergleich: Verschiedene Metriken}

Diese Studie vergleicht verschiedene Synthese-Metriken:

\begin{itemize}
  \item \textbf{Einfache Metriken}: Basierend auf einfachen Formeln (schnell, aber weniger genau)
  \item \textbf{Erweiterte Metriken}: Basierend auf komplexen Modellen (langsamer, aber genauer)
  \item \textbf{ML-basierte Metriken}: Basierend auf Machine-Learning-Modellen (sehr genau, aber komplex)
\end{itemize}

