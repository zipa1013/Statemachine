\chapter{Vergleich verschiedener Architektur-Paradigmen}\label{chap:vergleich_architekturen}

\noindent
Dieses Kapitel vergleicht verschiedene Architektur-Paradigmen für E/E-Architekturen und analysiert deren Vor- und Nachteile im Kontext der Simulation.

\section{Architektur-Paradigmen}

Verschiedene Architektur-Paradigmen werden in modernen Fahrzeugen verwendet:

\subsection{Distributed Architecture}

Traditionelle verteilte Architektur mit vielen ECUs:

\begin{itemize}
  \item \textbf{Struktur}: Viele spezialisierte ECUs, jeweils für spezifische Funktionen
  \item \textbf{Kommunikation}: CAN, LIN, FlexRay
  \item \textbf{Vorteile}: Bewährt, einfach, kostengünstig
  \item \textbf{Nachteile}: Hohe Komplexität, viele Kabel, schwierige Software-Updates
\end{itemize}

\subsection{Zonale Architecture}

Moderne zonale Architektur mit zentralen Rechenknoten:

\begin{itemize}
  \item \textbf{Struktur}: Wenige zentrale Rechenknoten, Zonen-Controller als Gateways
  \item \textbf{Kommunikation}: TSN-Ethernet-Backbone, CAN/LIN in Zonen
  \item \textbf{Vorteile}: Skalierbar, software-definiert, einfache Updates
  \item \textbf{Nachteile}: Komplexere Software, höhere Anforderungen an Rechenleistung
\end{itemize}

\subsection{Hybrid Architecture}

Kombination aus distributiver und zonaler Architektur:

\begin{itemize}
  \item \textbf{Struktur}: Zentrale Rechenknoten für neue Funktionen, Legacy-ECUs für bestehende Funktionen
  \item \textbf{Kommunikation}: TSN-Ethernet für neue Funktionen, CAN/LIN für Legacy
  \item \textbf{Vorteile}: Migration von Legacy-Systemen, schrittweise Modernisierung
  \item \textbf{Nachteile}: Komplexität durch zwei Architektur-Paradigmen
\end{itemize}

\section{Vergleichsanalyse}

Verschiedene Kriterien werden für den Vergleich verwendet:

\subsection{Komplexität}

\begin{table}[h]
  \centering
  \caption{Vergleich: Komplexität}
  \begin{tabular}{llll}
    \toprule
    Kriterium & Distributed & Zonal & Hybrid \\
    \midrule
    Anzahl ECUs & Hoch & Niedrig & Mittel \\
    Kabel-Komplexität & Hoch & Niedrig & Mittel \\
    Software-Komplexität & Niedrig & Hoch & Mittel \\
    Netzwerk-Komplexität & Niedrig & Hoch & Mittel \\
    \bottomrule
  \end{tabular}
  \label{tab:komplexitaet_vergleich}
\end{table}

\subsection{Kosten}

\begin{table}[h]
  \centering
  \caption{Vergleich: Kosten}
  \begin{tabular}{llll}
    \toprule
    Kriterium & Distributed & Zonal & Hybrid \\
    \midrule
    Hardware-Kosten & Mittel & Niedrig & Mittel \\
    Software-Kosten & Niedrig & Hoch & Mittel \\
    Wartungskosten & Hoch & Niedrig & Mittel \\
    Entwicklungskosten & Niedrig & Hoch & Mittel \\
    \bottomrule
  \end{tabular}
  \label{tab:kosten_vergleich}
\end{table}

\subsection{Performance}

\begin{table}[h]
  \centering
  \caption{Vergleich: Performance}
  \begin{tabular}{llll}
    \toprule
    Kriterium & Distributed & Zonal & Hybrid \\
    \midrule
    E2E-Latenz & Mittel & Niedrig & Mittel \\
    Skalierbarkeit & Niedrig & Hoch & Mittel \\
    Energieeffizienz & Niedrig & Hoch & Mittel \\
    \bottomrule
  \end{tabular}
  \label{tab:performance_vergleich}
\end{table}

\section{Erweiterte Vergleichs-Analysen}

Dieser Abschnitt präsentiert erweiterte Analysen für verschiedene Aspekte.

\subsection{Energieeffizienz-Vergleich}

\begin{table}[h]
  \centering
  \caption{Vergleich: Energieeffizienz}
  \begin{tabular}{llll}
    \toprule
    Architektur-Typ & Energieverbrauch & Optimierungs-Potenzial & Anmerkung \\
    \midrule
    Distributed & Hoch & Niedrig & Viele ECUs, hoher Overhead \\
    Zonal & Niedrig & Hoch & Zentrale Optimierung möglich \\
    Hybrid & Mittel & Mittel & Abhängig von Legacy-Anteil \\
    \bottomrule
  \end{tabular}
  \label{tab:energieeffizienz_vergleich}
\end{table}

\subsection{Wartbarkeits-Vergleich}

\begin{table}[h]
  \centering
  \caption{Vergleich: Wartbarkeit}
  \begin{tabular}{llll}
    \toprule
    Architektur-Typ & Software-Updates & Hardware-Updates & Diagnose \\
    \midrule
    Distributed & Schwierig & Einfach & Komplex \\
    Zonal & Einfach & Schwierig & Einfach \\
    Hybrid & Mittel & Mittel & Mittel \\
    \bottomrule
  \end{tabular}
  \label{tab:wartbarkeit_vergleich}
\end{table}

\section{Zusammenfassung}

Dieses Kapitel hat verschiedene Architektur-Paradigmen verglichen. Die zonale Architektur bietet Vorteile in Bezug auf Skalierbarkeit, Software-Updates und Energieeffizienz, während die distributed Architektur bewährt und kostengünstig ist. Die Hybrid-Architektur ermöglicht eine schrittweise Migration.


