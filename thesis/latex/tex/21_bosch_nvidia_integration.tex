\chapter{Bosch-Sensorik und NVIDIA DRIVE Thor Integration}\label{chap:bosch_nvidia}

\noindent
Dieses Kapitel beschreibt detailliert die Integration von Bosch-Sensorik und NVIDIA DRIVE Thor in moderne E/E-Architekturen. Die Kombination dieser Technologien ermöglicht hochperformante, zuverlässige Systeme für automatisiertes Fahren.

\section{Bosch-Sensorik Portfolio}

Bosch bietet ein umfassendes Portfolio an Sensoren für automatisiertes Fahren, das kontinuierlich weiterentwickelt wird.

\subsection{Bosch 8MP Multifunktionskamera}

Die Bosch 8MP Multifunktionskamera ist die neueste Generation von Kameras für automatisiertes Fahren:

\subsubsection{Technische Spezifikationen}

\begin{table}[h]
  \centering
  \caption{Technische Spezifikationen: Bosch 8MP Multifunktionskamera}
  \begin{tabular}{ll}
    \toprule
    Parameter & Wert \\
    \midrule
    Auflösung & 3840 × 2160 Pixel (8 Megapixel) \\
    Horizontales Sichtfeld & 120 Grad \\
    Vertikales Sichtfeld & 65 Grad \\
    Erkennungsreichweite & Bis zu 300 Meter \\
    Framerate & 30 fps (33.3 ms Periodizität) \\
    Interface & Ethernet 2.5 Gbps \\
    Kompression & H.264/H.265, optional RAW \\
    ASIL-Level & ASIL-B (ADAS), ASIL-D möglich \\
    Power-Verbrauch & 8 W typisch, 12 W maximal \\
    Betriebstemperatur & -40°C bis +85°C \\
    Serienproduktion & Geplant für 2026 \\
    \bottomrule
  \end{tabular}
  \label{tab:bosch_8mp_specs}
\end{table}

\subsubsection{Funktionen}

Die Bosch 8MP Multifunktionskamera unterstützt verschiedene ADAS-Funktionen:

\begin{itemize}
  \item \textbf{Adaptive Geschwindigkeits- und Abstandsregelung (ACC)}:
    \begin{itemize}
      \item Automatische Anpassung der Geschwindigkeit an vorausfahrende Fahrzeuge
      \item Abstandsmessung mit hoher Präzision durch 8MP Auflösung
      \item Erkennung bis zu 300 m Reichweite
    \end{itemize}
  
  \item \textbf{Notbremsungen innerhalb der eigenen Spur (AEB)}:
    \begin{itemize}
      \item Automatische Notbremsung bei erkannten Hindernissen
      \item Früherkennung durch hohe Auflösung und lange Reichweite
      \item ASIL-D für sicherheitskritische Anwendungen
    \end{itemize}
  
  \item \textbf{Spurhalten in städtischen Gebieten (LKA)}:
    \begin{itemize}
      \item Automatisches Spurhalten durch Markierungserkennung
      \item Breites Sichtfeld (120°) für enge Kurven
      \item Funktion auch bei schlechten Sichtverhältnissen
    \end{itemize}
  
  \item \textbf{Erkennung und Anhalten an roten Ampeln (TSR)}:
    \begin{itemize}
      \item Automatische Erkennung von Verkehrszeichen und Ampeln
      \item Präzise Erkennung durch hohe Auflösung
      \item Integration in Navigationssysteme
    \end{itemize}
\end{itemize}

\subsubsection{Integration in E/E-Architekturen}

Die Bosch 8MP Multifunktionskamera wird in E/E-Architekturen wie folgt integriert:

\begin{itemize}
  \item \textbf{Netzwerk-Integration}:
    \begin{itemize}
      \item Direkte Anbindung an Zonen-Controller oder Central Compute über Ethernet 2.5G
      \item TSN-Unterstützung für deterministische Übertragung
      \item Priorität P6 für sicherheitskritische Daten
    \end{itemize}
  
  \item \textbf{Software-Integration}:
    \begin{itemize}
      \item Standardisierte APIs für Kamera-Zugriff
      \item Unterstützung für verschiedene Bildverarbeitungsalgorithmen
      \item Integration in Sensorfusion-Systeme
    \end{itemize}
  
  \item \textbf{Power-Management}:
    \begin{itemize}
      \item Niedriger Power-Verbrauch für Energieeffizienz
      \item Wake-up-Mechanismen für schnelle Aktivierung
      \item Power-State-Management für optimale Energieverteilung
    \end{itemize}
\end{itemize}

\subsection{Bosch Radar-Sensoren}

Bosch bietet ein umfassendes Portfolio an Radar-Sensoren für verschiedene Anwendungen:

\subsubsection{Long-Range-Radar}

\begin{table}[h]
  \centering
  \caption{Technische Spezifikationen: Bosch Long-Range-Radar}
  \begin{tabular}{ll}
    \toprule
    Parameter & Wert \\
    \midrule
    Reichweite & Bis zu 250 Meter \\
    Geschwindigkeitsbereich & -200 km/h bis +200 km/h \\
    Winkelbereich & ±60 Grad \\
    Update-Rate & 20 Hz (50 ms) \\
    Auflösung & 0.5 m (Radial), 1.5° (Azimut) \\
    Interface & CAN-FD 2 Mbps oder Ethernet \\
    ASIL-Level & ASIL-B bis ASIL-D \\
    Power-Verbrauch & 3 W typisch, 5 W maximal \\
    \bottomrule
  \end{tabular}
  \label{tab:bosch_lrr_specs}
\end{table}

\subsubsection{Mid-Range-Radar}

\begin{table}[h]
  \centering
  \caption{Technische Spezifikationen: Bosch Mid-Range-Radar}
  \begin{tabular}{ll}
    \toprule
    Parameter & Wert \\
    \midrule
    Reichweite & Bis zu 160 Meter \\
    Geschwindigkeitsbereich & -150 km/h bis +150 km/h \\
    Winkelbereich & ±75 Grad \\
    Update-Rate & 25 Hz (40 ms) \\
    Auflösung & 0.3 m (Radial), 1.0° (Azimut) \\
    Interface & CAN-FD 2 Mbps oder Ethernet \\
    ASIL-Level & ASIL-B \\
    Power-Verbrauch & 2.5 W typisch, 4 W maximal \\
    \bottomrule
  \end{tabular}
  \label{tab:bosch_mrr_specs}
\end{table}

\subsubsection{4D-Imaging-Radar}

Das 4D-Imaging-Radar bietet zusätzlich zur 3D-Position auch Geschwindigkeitsinformationen:

\begin{itemize}
  \item \textbf{Reichweite}: Bis zu 200 Meter
  \item \textbf{Auflösung}: Hohe Auflösung für präzise Objekterkennung
  \item \textbf{Geschwindigkeit}: Präzise Geschwindigkeitsmessung
  \item \textbf{Anwendung}: Für hochautomatisiertes Fahren (L4/L5)
\end{itemize}

\subsection{Bosch LiDAR-Sensoren}

Bosch entwickelt High-Resolution-LiDAR-Sensoren für hochautomatisiertes Fahren:

\subsubsection{Technische Spezifikationen}

\begin{table}[h]
  \centering
  \caption{Technische Spezifikationen: Bosch High-Resolution-LiDAR}
  \begin{tabular}{ll}
    \toprule
    Parameter & Wert \\
    \midrule
    Layer & Bis zu 64 Layer \\
    Reichweite & Bis zu 200 Meter \\
    Punktdichte & Hoch (für präzise Objekterkennung) \\
    Winkelbereich & 360° (Rotating) oder 120° (Solid-State) \\
    Update-Rate & 10-20 Hz \\
    Interface & Ethernet 10 Gbps \\
    ASIL-Level & ASIL-B \\
    Power-Verbrauch & 15 W typisch, 25 W maximal \\
    \bottomrule
  \end{tabular}
  \label{tab:bosch_lidar_specs}
\end{table}

\section{NVIDIA DRIVE Thor}

NVIDIA DRIVE Thor ist die neueste Generation der NVIDIA DRIVE Plattform für autonomes Fahren.

\subsection{Technische Spezifikationen}

\begin{table}[h]
  \centering
  \caption{Technische Spezifikationen: NVIDIA DRIVE Thor}
  \begin{tabular}{ll}
    \toprule
    Parameter & Wert \\
    \midrule
    GPU-Performance & 2000 TOPS (Tera Operations Per Second) \\
    CPU & NVIDIA Grace CPU (ARM Neoverse V2) \\
    CPU-Kerne & 12 Kerne @ 3.0 GHz \\
    RAM & 512 GB LPDDR5X \\
    Storage & 1 TB NVMe SSD \\
    Ethernet-Ports & Multiple (2.5G, 10G) \\
    CAN-FD-Ports & Multiple (2 Mbps) \\
    ASIL-Level & ASIL-D zertifiziert \\
    Power-Verbrauch & 80 W typisch, 150 W maximal \\
    Betriebstemperatur & -40°C bis +85°C \\
    Software & NVIDIA DRIVE OS \\
    \bottomrule
  \end{tabular}
  \label{tab:nvidia_drive_thor_specs}
\end{table}

\subsection{GPU-Architektur}

Die NVIDIA DRIVE Thor GPU ist spezialisiert auf KI/ML-Workloads:

\begin{itemize}
  \item \textbf{Tensor-Cores}: Spezialisierte Einheiten für Tensor-Operationen
  \item \textbf{RT-Cores}: Ray-Tracing-Cores für visuelle Anwendungen
  \item \textbf{Multi-Instance-GPU (MIG)}: Partitionierung für Multi-Domain-Isolation
  \item \textbf{Unified Memory}: Effiziente CPU-GPU-Zusammenarbeit
\end{itemize}

\subsection{Software-Stack}

NVIDIA DRIVE OS bietet einen vollständigen Software-Stack:

\begin{itemize}
  \item \textbf{Betriebssystem}: Linux-basiert mit Echtzeit-Erweiterungen
  \item \textbf{Middleware}: AUTOSAR Adaptive, DDS, SOME/IP
  \item \textbf{KI-Frameworks}: TensorRT, CUDA, cuDNN
  \item \textbf{Entwicklungstools}: NVIDIA DRIVE SDK, Simulator
\end{itemize}

\section{Integration von Bosch-Sensoren mit NVIDIA DRIVE Thor}

Die Integration von Bosch-Sensoren mit NVIDIA DRIVE Thor ermöglicht hochperformante Perzeption:

\subsection{Sensor-Integration}

\begin{itemize}
  \item \textbf{Bosch 8MP Kamera}:
    \begin{itemize}
      \item Direkte Anbindung über Ethernet 2.5G
      \item Unterstützung für bis zu 12 Kameras gleichzeitig
      \item Echtzeit-Verarbeitung auf DRIVE Thor GPU
    \end{itemize}
  
  \item \textbf{Bosch Radar}:
    \begin{itemize}
      \item Integration über CAN-FD oder Ethernet
      \item Unterstützung für bis zu 9 Radare
      \item Sensorfusion auf DRIVE Thor
    \end{itemize}
  
  \item \textbf{Bosch LiDAR}:
    \begin{itemize}
      \item Direkte Anbindung über Ethernet 10G
      \item Unterstützung für bis zu 4 LiDAR-Sensoren
      \item Punktwolken-Verarbeitung auf DRIVE Thor GPU
    \end{itemize}
\end{itemize}

\subsection{Performance-Benchmarks}

Die Kombination von Bosch-Sensoren mit NVIDIA DRIVE Thor bietet hervorragende Performance:

\begin{table}[h]
  \centering
  \caption{Performance-Benchmarks: Bosch-Sensoren mit NVIDIA DRIVE Thor}
  \begin{tabular}{llll}
    \toprule
    Sensor & Modell & Inferenz-Zeit & GPU-Last \\
    \midrule
    Bosch 8MP Kamera & YOLOv8 & 15 ms & 0.975\% \\
    Bosch 8MP Kamera (8x) & YOLOv8 & 120 ms & 7.8\% \\
    Bosch Radar & Radar-Fusion & 5 ms & 0.1\% \\
    Bosch LiDAR & Point-Cloud-Processing & 20 ms & 1.5\% \\
    \bottomrule
  \end{tabular}
  \label{tab:bosch_nvidia_performance}
\end{table}

\section{Zusammenfassung}

Dieses Kapitel hat die Integration von Bosch-Sensorik und NVIDIA DRIVE Thor detailliert beschrieben. Die Kombination dieser Technologien ermöglicht hochperformante, zuverlässige Systeme für automatisiertes Fahren mit exzellenter Perzeptions-Performance und niedrigen Latenzen.

